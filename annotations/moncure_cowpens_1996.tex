%\red{Nicholson}
%1996. The author holds a M.A. and Ph.D.  in Modern European History from Cornell
%University, and is a former professor of history at the U.S.  Military Academy.
%The author is a member of the United States Army, so the source is bias in favor
%of the American Side. This book was useful in determining a timeline and key
%decisions for the operational situation section. I highly recommend this source.
%
%\red{Flores}
%Moncure was a professor of military history at Davidson College, and also is a
%European historian who holds a M.A. and Ph.D. in Modern European History from
%Cornell University. Moncure's book was good in the sense that it gave different
%perspective of the battles in forms of  letters from different individuals,
%from privates to great leaders.
%
%\red{Futch}
%The author holds a M.A. and Ph.D. in Modern European History from Cornell
%University, and is a former professor of history at the U.S. Military Academy.
%This handbook was a useful resource; breaking the battle down with maps of each
%phase of combat and providing detailed analysis of the courses of action taken
%by the commanders.
%
%\red{Bieber}
%The author is a former US Army officer who holds an M.A. and PhD in Modern
%European History from Cornell University.  This publication approaches the
%Battle of Cowpens from the perspective of using both the text and location as
%ground for a Staff Ride.  The collection of letters and personal histories in
%the back of the book serve as an invaluable tool in analyzing the participants
%emotions and recollections of the Battle.
%
The author holds a M.A. and Ph.D.  in Modern European History from Cornell
University, and taught history at the U.S. Military Academy and Davidson
College. This book was useful in determining a timeline and key decisions for
the operational situation section. It provided individual perspectives from the
battles via letters from primary sources. It also broke down the combat
operations with maps by phase and provided a detailed analysis of the courses of
action taken by the commanders.
