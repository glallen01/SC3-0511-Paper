\subsection{Comparison of the Antagonists}

At the beginning of the war the objective of Great Britain was to keep control
over the Colonies in America \cite[126]{ladenburg_causes_1989}
\cite[45]{stewart_2005}.  The objective of the 13 Colonies in North America was
to loosen and remove control from Great Britain.

Great Britain’s Parliament passed several acts regarding the Colonies.  The acts
restricted territory expansion, regulated trade, imposed taxes, and punished
disobedience.  Although some of the acts were repealed and some never came to
power, the British Parliament kept passing them despite a revolt that was too
obvious in the 13 Colonies \cite[90,107-109]{ladenburg_causes_1989}.  In a way, Great Britain
wanted to prove to everyone that it had the autocratic right to govern the
Colonies \cite[126]{ladenburg_causes_1989}.

The political representatives of the 13 Colonies were elected by the people.
Colonies enjoyed a kind of political autonomy by passing laws that were not
counter-British laws \cite[p.71]{ladenburg_causes_1989}.  They did not have representatives in
the British Parliament and opposed the laws prohibiting their right to be
represented in Parliament \cite[p.91]{ladenburg_causes_1989}.  They even questioned the right of
Great Britain to govern them \cite[p.96]{ladenburg_causes_1989}.

After the French and Indian War (1754-1763) the 13 Colonies felt victorious.
They eliminated the threat of their most dangerous enemy France.  They had the
opportunity for territorial expansion into Native American lands and Canada
whose control was taken over by the British after the war.  Before the war
started Great Britain enticed the people from the 13 Colonies to settle in the
western land.  After the war, Britain changed its policy and restricted
movement.  They justified the restriction as a measure to prevent hostilities
among the people from the 13 Colonies and the new territories.  In the French
and Indian War the American Colonies participated with a large militia force.
After the war they believed they could defend their territory with their own
forces.  Instead, Great Britain increased military forces in the 13 Colonies.
The Colonials had the opinion that Britain was ready to enforce British laws by
using weapons.  Great Britain wanted to stop smuggling in the 13 Colonies
through any means possible, including using force.  Most of the British laws for
the colonies consisted of imposing taxes and restricting trade.  Peacefully
accepting the taxes would have meant that more and more taxes could be imposed.
Great Britain justified the taxes with the following ideas: the people from
Great Britain were paying much higher taxes, some parts of the territory didn’t
have representatives in the Parliament but were paying the taxes, and the taxes
should cover the expenses of Great Britain from defending the Colonies.
Restricting trade was painful for the 13 Colonies.  They were forced to export
raw materials to Great Britain and import final products from Great Britain.
Because of this they could not develop their own industry although they did
possess the potential to do so.  The Colonies were restricted on trade with
other countries even though the trade conditions were much better than those
with Great Britain.  Great Britain defended the restriction of trading with
other countries by saying that doing so would prevent the other countries to
grow.  But, at the same time, Great Britain feared the growth of power and
wealth of its Colonies and decline of itself.  Most of the people came to
America for economic prospects and political and religious freedoms.  The
greatest obstacle for its achievement before the Revolutionary War was Great
Britain
\cite[76,81-91]{ladenburg_causes_1989}\cite[133-134]{knollenberg_growth_2003}.

During the Revolutionary War, the national objective of the 13 Colonies
transitioned to a fight for total independence from Great Britain.  The best
evidence is the Declaration of Independence brought on July 4, 1776
\cite[119]{ladenburg_causes_1989}.  The national objective of Great Britain
focused on keeping the 13 Colonies in the British Empire.

After 1765, people in the 13 Colonies began forming groups to fight for their
rights.  The groups were merging into bigger organizations.  The members were
called “Sons of Liberty” \cite[p.227,234]{knollenberg_growth_2003}.  They initiated a series of
protests throughout the colonies.  In Boston in 1773, the Sons of Liberty took
action by throwing tea from a British ship into the water, known today as the
Boston Tea Party.  Great Britain punished them by dissolving the local
governments in the Massachusetts Bay Colony, appointing a military general as
Governor, and imposing severe laws \cite[p.108-109]{ladenburg_causes_1989}.  In 1774, the First
Continental Congress was held. The Continental Congress unsuccessfully
petitioned the King of Great Britain, George III, to redress grievances and to
recognize the rights of the colonies \cite[p.9]{knollenberg_growth_2003}.  After the First
Continental Congress, committees were established throughout the colonies which
would become official governments in the near future \cite[p.46]{stewart_2005}.

At the same time, the Colonies were preparing for a possible war by forming
military forces and looking for allies.  They “took over control of the militia
and other colonial military resources such as armories and powder stores.  They
also identified which local militia officers could be trusted and which were
known to be loyal to Britain” \cite[p.46]{stewart_2005}.   The Colonies formed and trained
militia companies called Minutemen because they were supposed to be capable of
“reacting in a minute”.  After the Second Continental Congress in 1775, the
Colonies formed the Continental Army.  They negotiated with France, a rival and
enemy of Great Britain.  A formal military alliance was formed with The Treaty
of Alliance in 1778, even though France was already aiding America with supplies
for the war \cite[p.46,51,76,85]{stewart_2005}.

The American Revolutionary War began with the Battles of Concord and Lexington
in 1775 \cite[p.111-113]{ladenburg_causes_1989}.  The same year, the 13 Colonies tried to take
control over it after the Siege of Boston \cite[p.55]{stewart_2005}.  The Continentals
implemented a strategy that made the British divide their forces, cutting them
off from supplies and attacking them when they were weakened.  This allowed the
Rebels to maintain and avoid the destruction of the Continental Army. 

Great Britain relied on all of its forces, especially the Royal Navy.  American
colonists loyal to the British King (Loyalists) organized militia, but without
any notable success.  The British hired German mercenaries to help them in the
war \cite[p.133]{higginbotham_daniel_1961}.  Their purpose was to dissuade the
colonies from rebellion and acknowledge Great Britain as their ruler.  The
initial strategy was to force pressure on the colonies by blocking and raiding
the American coast. Britain expected support from the Loyalists and the Native
American tribes within the colonies. The British did not want to take a risk and
lose a decisive battle, or win many indecisive battles
\cite[p.548]{mackesy1962british}.  Although the British assumed control of
several American ports it did not have much effect on the rebels.  Britain
thought it would end the revolution if it demonstrated power in an attack
\cite[p.49]{stewart_2005}.  The British started offensive operations from Canada
and occupied New York, cutting off the territory of New England that they
considered to be the center of the American Revolution
\cite[p.64-70]{stewart_2005}.  The idea was to isolate the center of the
rebellion, New England, in order to annul the effect of the rebellion in all of
the colonies \cite[p.548]{mackesy1962british}.  The Americans were persistent.
They won a decisive victory at the Battles of Saratoga in 1777, which ended the
British strategy in the north and opened the doors for France to enter the war
in support of the American Colonies \cite[85]{stewart_2005}.  Britain adjusted
its strategy to conquering the south.  There it again expected help from the
Loyalists who were supposed to establish local ruling governments.  In the
beginning, the British forces succeeded in taking control of much of the
territories in Georgia and South Carolina.  They continued to North Carolina,
but with weakened forces.  At the same time they were forced to fight constantly
against the American militia and couldn’t keep the southern territories under
control. \cite[551]{mackesy1962british}

Table \ref{table:comparison} is a comparison of diplomatic, informational, military,
and economic power of the opposing nations:


%\documentclass[letterpaper,12pt]{article}
%\usepackage[margin=1in]{geometry}
%\usepackage{xtab}
%\usepackage{ctable}
%\usepackage{enumitem}

% Overview text goes here.
%%\begin{document}
%  \ctable[
%  	cap=foo,
%	caption=Elements of National Power,
%	]{l>{\tiny}p{2.7in}>{\tiny}p{2.7in}}{}{\FL
%%\begin{table}[h]
\begin{singlespace}
\footnotesize
  \tablecaption{Comparison of the antagonists' elements of national power.}
  %\tablecaption{Comparison of the Antagonists}
  \tablefirsthead{\FL\centering{USA} & \centering{Great Britain}\ML}
  \tabletail{\ML\multicolumn{2}{c}{\centering{continued\ldots}}\ML}
  \tablelasttail{}
  \tablehead{     \multicolumn{2}{c}{Comparison of the Antagonists, (cont)}\NN\centering{USA} & \centering{Great Britain}\NN}
  \label{table:comparison}
\begin{mpxtabular}{p{3in}p{3in}}%\FL
  %\centering{USA} & \centering{Great Britain} \ML
  \multicolumn{2}{c}{Diplomatic}\ML
  \begin{itemize}[nolistsep,leftmargin=*]
    \item The Treaty of Alliance with France was formed in 1778
    \item Spain declared war against Great Britain in 1779 
    \item Dutch Republic entered the war against Great Britain in 1780 \cite[85,89]{stewart_2005}
  \end{itemize}
  &
  \begin{itemize}[nolistsep,leftmargin=*]
      \item  Did not have allies
      \item  Several German states provided troops of
	mercenaries \cite[62]{stewart_2005}
      \item  Retained loyalty of some colonists
  \end{itemize}\ML
  \multicolumn{2}{c}{Informational}\ML
	\begin{itemize}[nolistsep,leftmargin=*]
	    \item  Colonies published information in 27 newspapers; pamphlets
	      etc. \cite[49,199]{knollenberg_growth_2003}
	    \item  They used propaganda.  Examples: ``Farmer's Letters'' by John
	      Dickinson published in 1767-1768 throughout the colonies; “Boston
	      Massacre” engraving by Paul Revere, and public orations from 1771
	      to 1775; the pamphlet “Common Sense” by Thomas Paine in 1776;
	      Liberty poles
	      erected \cite[3,48-53,81,227]{knollenberg_growth_2003} \cite[101]{ladenburg_causes_1989}
	    \item  Communication was well established, and many letters were
	      used as means of communication \cite[167,264]{knollenberg_growth_2003}
	    \item  They did not have many difficulties in providing information
	      about the enemy \cite[63]{ladenburg_causes_1989}
	\end{itemize}
	&
	\begin{itemize}[nolistsep,leftmargin=*]
	    \item  Used publications throughout British Empire
	    \item  Used propaganda, like the printing of  “A New Method of
	      Macaroni Making as Practiced at Boston in North America”
	    \item  Sometimes the propaganda was confusing, and could be
	      interpreted as counter-propaganda, like the printings “The
	      Bostonians in Distress”, or “The Wise Men of Gotham and Their
	      Goose” \cite[168-169]{knollenberg_growth_2003}
	    \item  Letters were used as means of communication \cite[169]{knollenberg_growth_2003}
	    \item  Communication faced problems of great distance from Great Britain and hostile territory
	    \item  Had difficulties in providing intelligence information \cite[545]{mackesy1962british}
	\end{itemize}\ML
	\multicolumn{2}{c}{Military}\ML
	\begin{itemize}[nolistsep,leftmargin=*]
	    \item  Had significant allies in European great powers: France,
	      Spain, and Dutch Republic \cite[p.89]{stewart_2005} 
	    \item  At the beginning of the war had militia as armed forces for
	      local defense \cite[p.56]{stewart_2005} 
	    \item  People from militia were not used to fighting on a wide area
	      \cite[p.48]{pancake_1985} 
	    \item  Organized and trained companies of militia for rapid response
	      called minutemen \cite[p.46]{stewart_2005} 
	    \item  Formed Continental Army, but militia men were constantly
	      filling the gaps \cite[p.53]{stewart_2005} 
	    \item  The leaders had experience in waging war, but more as
	      tactical than strategic leaders
	      \cite[7]{higginbotham_daniel_1961} \cite[p.51]{stewart_2005} 
	    \item  Officers were elected \cite[p.51]{pancake_1985}, but the Continental Army faced the problem of not having enough men competent to be officers 
	    \item  The soldiers were lacking training and discipline
	      \cite[p.52]{pancake_1985} 
	    \item  People who were wealthier than the others avoided recruitment
	      \cite[p.26]{stephenson_patriot_2007}, and even the criminals were
	      recruited \cite[p.27]{stephenson_patriot_2007} 
	    \item  People were deserting from the military forces
	      \cite[p.48]{pancake_1985} 
	    \item  The number of men engaged in the war was decreasing because
	      of diseases \cite[p. 33]{stephenson_patriot_2007} 
	    \item  Had problems with supplies
	      \cite[p.105]{stephenson_patriot_2007}
	\end{itemize}
	&
	\begin{itemize}[nolistsep,leftmargin=*]
	    \item  Didn’t have allies, but German mercenaries, and American
	      Loyalists were fighting on their side
	      \cite[p.133]{higginbotham_daniel_1961} 
	    \item  Had standing armed forces
	      \cite[p.48]{higginbotham_daniel_1961} 
	    \item  British forces were far from home 
	    \item  Had the greatest naval power in the world
	      \cite[p.43]{stephenson_patriot_2007} 
	    \item  Had problems with the choice of officers
	      \cite[p.543-544]{mackesy1962british} 
	    \item  The soldiers were well-trained and well-disciplined
	      \cite[p.42]{pancake_1985} 
	    \item  They did not fight for 15 years prior the war and some of
	      them were inexperienced \cite[p.44-45]{stephenson_patriot_2007} 
	    \item  Most of the recruited were poor
	      \cite[p.38]{stephenson_patriot_2007}, many of them criminals
	      \cite[p.36]{stephenson_patriot_2007} 
	    \item  Faced the problem of deserting of men from the military
	      forces \cite[p.48]{pancake_1985} 
	    \item  The number of men engaged in the war was decreasing because
	      of diseases \cite[p.62]{stewart_2005} 
	    \item  They held the strategic initiative
	      \cite[p.540]{mackesy1962british} 
	    \item  Had problems with supplies, especially because of the
	      transport \cite[p.541]{mackesy1962british}\cite[62]{stewart_2005} 
	    \item  They could not get reinforcements immediately if they needed it
	\end{itemize}\ML
	\multicolumn{2}{c}{Economic}\ML
	\begin{itemize}[nolistsep,leftmargin=*]
	     \item Had economic growth before the Revolution
	       \cite[77,91]{ladenburg_causes_1989} but at the time of Revolution the growth declined
	     \item In 1777, Congress had a debt of \$20,000,000
	       \cite[p.102]{stephenson_patriot_2007}
	       \cite[34]{higginbotham_daniel_1961}
	     \item Manufacturing was restricted
	       \cite[p.76]{ladenburg_causes_1989}
	     \item Most of the economy was based on plantations, hunting,
	       fishing, exploring the forests \cite[p.75]{ladenburg_causes_1989} 
	     \item They were restricted in trading by Great Britain, but many
	       were not obeying the laws; they even negotiated secretly for
	       trading \cite[p.229]{higginbotham_daniel_1961}
	     \item Issued paper money but its value  was dropping during the war
	       \cite[p.27]{stephenson_patriot_2007}
	     \item Did not have efficient federal tax system to collect money 
	     \item France helped American economy by loaning large sums of money
	       \cite[p.233]{higginbotham_daniel_1961}
	\end{itemize}
	&
	\begin{itemize}[nolistsep,leftmargin=*]
	    \item Had a national debt of £140,000,000 after the French and
	      Indian War\cite[p.88]{ladenburg_causes_1989} 
	    \item Industrial Revolution had begun \cite[p.37]{pancake_1985}, but
	      people were losing their jobs \cite[p.37]{stephenson_patriot_2007} 
	    \item Explored their colonies to increase their wealth
	      \cite[p.75]{ladenburg_causes_1989} 
	    \item Established trade regulations to protect their economy
	      \cite[p.76]{ladenburg_causes_1989} 
	    \item Had banks, national currency, and well established paying systems 
	    \item Had tax system that collected money 
	\end{itemize}\ML
      \end{mpxtabular}
\end{singlespace}
%\end{table}
%\end{document}



British forces hoped that Americans loyal to the King would establish
governments where they won the battles. However, Loyalists were outnumbered by
the Patriots \cite[p.201]{knollenberg_growth_2003}.  Britain could not govern the colonies.
British forces were too few to occupy every necessary territory in the 13
Colonies.  They could not even follow American forces that were retreating from
the large territories \cite[p.540-541]{mackesy1962british}.  Britain wanted to win a decisive
battle, but Americans engaged their military wherever and whenever possible.
There was no defined objective to attack or control that could decisively sway
the war in Britain’s favor.  After France, Spain, and the Dutch Republic entered
the war, Great Britain was engaged in a war across the world
\cite[540-541]{mackesy1962british}.  Americans were in their homeland.  It was a land far from Great
Britain, and too large to control from a distance.   The Americans’
revolutionary spirit continued to grow and thrive during this time.  Despite all
of the limits imposed on the Americans, it seemed nothing could stop them win
their war for independence. 

At the start of the conflict in April 1775, the 13 Colonies had only militia as
armed forces.  They were activated when there was a need for local action.  That
same year more than 15,000 militiamen participated in the Siege of Boston.  The
Second Continental Congress decided to form a regular army.  It was called the
Continental Army.  In 1775, the enlistment term for a soldier was just for the
current year.  Many non-trusting categories of people were not allowed to join
the Army.  After George Washington’s defeat in New York in 1776, the Continental
Army numbered less than 5,000 men.  Congress changed its policy and started
recruiting without taking previous restrictions into consideration. The
enlistment was for three years \cite[p.44-45]{pancake_1985}.  The age group of the
military personnel available was between 16 and 60.   In October 1778 there were
more than 18,000 men in the Army \cite[p.37]{pancake_1985}.  The number of soldiers in the
Continental Army constantly varied throughout each year of the war:  in 1775  it
averaged approximately 18,950 men; in February 1776 around 18,887;  in March
1776 it had decreased to 7,720;  in November 1776 numbered around 40,962; less
than 30,000 in March to 33,021 in October 1777; an average of 24,000 in 1778; an
average of 21,500 in 1779, and an average of 14,000 in 1780
\cite[32-33]{stephenson_patriot_2007}.  Most of the soldiers were young and poor.  The population of 13
Colonies in 1775 was about 2.5 million.  The estimated number of men fighting
during the Revolutionary War was about 200,000
\cite[p.30-31]{stephenson_patriot_2007}.  France,
Spain, and Dutch Republic allied with the American Colonies during the war.

Militia men were neither well-trained, nor well-disciplined.  Equipping these
men was considered expensive because they were much more wasteful with supplies
when compared to the Continental Army regulars.  Their service lasted for a
short period of six months or less. Regardless, they were crucial in helping the
Continental Army by weakening the British Army through a constant fight
\cite[p.52]{pancake_1985}.  The Continental Army was also faced with problems in training,
discipline, and supplies \cite[p.47]{pancake_1985}.

From the beginning till the end of the war Great Britain engaged standing
armies.  They were permanent formations who were paid for the service.  The
British also hired mercenaries from several German states throughout the Holy
Roman Empire.  During the war about 30,000 German mercenaries were fighting
alongside the British \cite[p.62]{stewart_2005}.  In 1750, the population of Great Britain
was approximately 5.7 million and by 1766 grew to about 8 million.  The British
Army had 48,500 men, but only about 7,000 in North America in 1775.  Its peak
was in 1778 when it consisted of 50,000 men \cite[37]{stephenson_patriot_2007}
\cite[1]{knollenberg_growth_2003}.  Each soldier’s enlistment lasted either three years or until the end of
the war.  The men were usually well-trained and well-disciplined.  In contrast
to the good training and discipline, there were problems with personnel and
supplies.  The possibility of corruption in the British government and its
military forces cannot be excluded.  Officer promotions only occurred if there
was a vacancy in the higher ranks.  Officer commissions were purchased and many
military officers had political careers.  The age group of the military
personnel available was between 15 and 70 \cite[p.39-43]{pancake_1985}.  In similarity
with the Americans, most of the British military men were young and poor.  At
the time of the war 45 percent of the British population was under the age of
20.  The greatest military power that Great Britain possessed was actually the
Royal Navy \cite[p.42-43]{stephenson_patriot_2007}.  As the war progressed, the Navy was more
engaged in waging war with France, Spain, and the Dutch Republic.  The British
Army became a core of the forces fighting against the American Colonists.  “Both
[Great Britain and USA] armies had tough fighters” \cite[p.55]{pancake_1985}. Despite the
problems faced by each entity, they were ready for war.

The first experience that Americans had with warfare was in the period from 1697
to 1689 against Native American Indians \cite[p.1]{peckham_1964}.  During the wars with
the Indians the main forces were militia men who were self-supplied,
periodically trained, and engaged locally when they were needed
\cite[p.30]{stewart_2005}.
In the period between 1689 and 1763 the main antagonists in Europe fought in
four great coalition wars.  Great Britain participated in all of them. Since
Americans were part of the British Empire they too were engaged in these wars.
“Americans and Europeans called these wars by different names.  The War of the
League of Augsburg (1689–1697) was known in America as King William’s War, the
War of Spanish Succession (1701–1713) as Queen Anne’s War, the War of Austrian
Succession (1744–1748) as King George’s War, and the final and decisive
conflict, the Seven Years’ War (1756–1763) as the French and Indian War”
\cite[p.32]{stewart_2005}.  Americans were not impressed with the British military
performance on land \cite[p.34]{stewart_2005}.  On the other hand, after the French and
Indian War and later war with Indians led by Pontiac, the British believed that
“the colonists could not defend themselves” \cite[p.87]{ladenburg_causes_1989}.  Great Britain
fought against France during all of the aforementioned wars.  After the French
and Indian war, Britain took over the colonies in North America from France
increasing the size of its Empire \cite[p.87]{ladenburg_causes_1989}.  Both Americans and British
came out of the wars as winners.

The fact that both the Americans and British faced a problem with deserters
during the American Revolutionary War clearly depicts that each warring faction
struggled to keep cohesiveness in their armed forces.  The Private Secretary to
British Admiral Lord Richard Howe, Ambrose Serle, published in his journal that
the American army was “the strangest that was ever collected: old men of 60,
boys of 14, and blacks of all ages” \cite[p.20]{stephenson_patriot_2007}.  On the other hand,
“one-third of Americans were patriot, one-third pro-British, and the remaining
third neutral or persuadable one way or the other”
\cite[p.52]{stephenson_patriot_2007}.  There
are different sources for the number of Loyalists, but the impression is that
they were less numbered than the Patriots
\cite[p.199-203]{knollenberg_growth_2003}.  Some
Americans were even relieved from military duty for various reasons such as
Quakers because of religious reason \cite[p.51]{pancake_1985}.

British Admiral Lord Richard Howe showed sentiment towards Americans. Loyalists
became angry with him because of his empathy \cite[p.60]{stephenson_patriot_2007}.  British
General Phillips said: “We cannot forget that when we strike, we wound a
brother” \cite[p.547]{mackesy1962british}.  The support for the war against Americans was not
unanimous in the British Parliament \cite[p.16]{pearson_failure_}, but still they waged war.

%----



%\subsubsection{National (Strategic) objectives}
%
%\index{Strategic Objectives}
%
%\paragraph{British}
%
%Strategic Objectives: ``maintain the thirteen colonies as a British
%possession''\cite[2]{moncure_cowpens_1996}
%
%``British military objectives were fourfold: separate the New England colonies
%from the others by seizing the Hudson River north to Lake Champlain; isolate the
%'bread basket' colonies of Pennsylvania and Maryland; control the southern
%populace by holding Charleston, Georgetown, and the line of the Santee River;
%and, finally, blockade the entire American coast to prevent an influx of arms
%from abroad.''\cite[2]{moncure_cowpens_1996}
%
%Strategic Strategies: ``divide the Colonies by applying economic sanctions to
%the most rebellious''\cite[2]{moncure_cowpens_1996}
%
%%Other British National objectives: other colonies, West Indies, trade routes,
%%(discuss mercantilism?), 
%
%%``\ldots d'Estaing's presence in the Caribbean together with a Spanish fleet in
%%Havana (after Spain's entry on the American side in June 1779) forced Clinton to
%%send 8,000 troops to the West Indies. These transfers weakened Clinton so
%%severely that he never seriously challenged Washington's position at West Point,
%%New York.''\footcite[10]{moncure_cowpens_1996} \index{d'Estaing}
%
%\paragraph{Colonies}
%
%\ldots
%
%\subsubsection{Instruments of National Power}
%
%%\begin{table}
%%  \begin{center}
%%  \begin{tabular}{lll}\toprule
%%    Instrument & British & American \\\midrule
%%	a & a & a \\\bottomrule
%%  \end{tabular}
%%  \end{center}
%%  \caption{Comparison of British and Colonial instruments of National Power.}
%%\end{table}
%
%\ldots
%
%\subsubsection{Military Systems}
%
%\ldots
%
%\subsubsection{Previous performance}
%
%\ldots
