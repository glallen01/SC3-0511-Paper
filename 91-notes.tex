\subsection{\emph{The American Revolution: A Global War}}
% 91-notes.tex
%%%%%%%%%%%%%%%%%%%%%%%%%%%%%%%%%%%%%%%%%%
% The American Revolution: A Global War  %
%%%%%%%%%%%%%%%%%%%%%%%%%%%%%%%%%%%%%%%%%%

%%  In 1756, Frederick II, the Great, of Prussia conducted a pre-emptive invasion
%%  of Saxony. He had been allied against by Austria, France, Russia, Sweden and
%%  Saxony, and a defensive alliance with Britain alone. With this attack Frederick
%%  ``succeeded only in solidifying the coalition against
%%  him,''\cite{dupuy_1977} and drew Britain as an ally into a general conflict
%%  between European powers where it would seek and achieve several strategic
%%  gains.
%%  
%%  Britain routed the French in Canada, India, several Carribean
%%  islands,\footnote{``In the Caribbean, British forces captured two French
%%  islands valuable for their sugar, Guadeloupe and Martinique, territories
%%  considered by many more crucial to the French Empire than all of vast, vacant
%%  Canada. They also took Grenada, St. Lucia, St. Vincent, and
%%  Dominica.''\cite[8]{dupuy_1977}} Gor\'ee, France's chief slave-trading center
%%  in Western Africa, as well as Gibraltar and Minorca.\cite[7-9]{dupuy_1977}
%%  From Spain, Britain seized Havana, Manila and the Philippines. Ultimately the
%%  Prussians were saved by the favor of Peter III when he succeeded Empress
%%  Elizabeth to the throne of Russia in 1762, rather than by the British, despite
%%  their support. As Dupuy el al describe, the European land campaigns of the
%%  Seven Years' War, fought over dynastic aims, had little lasting impact.
%%  However, the naval and foreign campaigns fought between Britain and France over
%%  empire and mercantileism, reshaped the sea-lanes in Britain's favor. In the
%%  Treaty of Paris France lost Canada and all land East of the Missippi
%%  River.\cite[10]{dupuy_1977}
%%  
%%  --- 
%%  
%%  \begin{quote}
%%  `John Adams felt strongly that Americans tended to confuse the American
%%  Revolution with the Revolutionary War. The war, he wrote to Thomas Jefferson,
%%  ``was no part of the Revolution. It was only an effect and consequence of it.''
%%  Adams, well aware that he was presenting and unconventional view, went on to
%%  state that the Revolution had taken place in the minds of the people from 1760
%%  to 1775, ``before a drop of blood was spilled at
%%  Lexington.'''\cite[15]{dupuy_1977}
%%  \end{quote}

\subsection{Cook Notes}

\input{cook-notes.txt}
