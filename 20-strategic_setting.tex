\section{The Strategic Setting}

\subsection{Causes of the Conflict}

%%% --begin-notes
\subsection{--notes--}\footcite[2]{moncure_cowpens_1996}

Enlightenment political theory:

French and Indian War (1754-63)

Village green in Lexington, MA 19 APR 1775

trade unrestricted prior to 1760s

King sends:\\
Major General Sir William Howe\\
Major General Sir Henry Clinton\\
Major General Sir John Burgoyne

\paragraph{--notes--}\footcite[]{}
``Shy. John, \emph{A People Numerous and Armed.} pp. 198-199 and 230-232. The British contemplated cutting loose New England and retaining the richer lands of the South in 1778. but never pursued a negot~ated settlement with the Continental Congress.''

%%% --end-notes



\subsection{Comparison of the Antagonists}

Overview text goes here.

\ctable[
	cap = Comparison of Antagonists,
	caption = Comparison of US and Colonial Forces,
	]{p{2.5in}p{2in}p{2in}}{
	\tnote[a]{note a}}
	{\toprule\small
Factor				&	Colonies	& British	\\\midrule
National (Strategic) objectives	&			&		\\
Insturments of National Power	&			&		\\
Military Systems		&			&		\\
Previous performance		& French and Indian War	&		\\\bottomrule
	}



\subsubsection{National (Strategic) objectives}
\index{Strategic Objectives}
\paragraph{British}

Strategic Objectives: ``maintain the thirteen colonies as a British possession''\footcite[2]{moncure_cowpens_1996}

``British military objectives were fourfold: separate the New England colonies from th eothers by seizing the Hudson River north to Lake Champlain; isolate the ''bread basket`` colonies of Pennsylvania and Maryland; control the southern populace by holding Charleston, Georgetown, and the line of the Santee River; and, finally, blockade the entire American coast to prevent an influx of arms from abroad.''\footcite[2]{moncure_cowpens_1996}

Strategic Strategies: ``divide the Colonies by applying economic sanctions to the most rebellious''\footcite[2]{moncure_cowpens_1996}

Other British National objectives: other colonies, West Indies, trade routes, (discuss mercentilism?), 

``\ldots d'Estaing's presence in the Caribbean together with a Spanish fleet in Havana (after Spain's entry on the American side in June 1779) forced Clinton to send 8,000 troops to the West Indes. These transfers weakened Clinton so severely that he never seriously challenged Washington's position at West Point, New York.''\footcite[10]{moncure_cowpens_1996}
\index{d'Estaing}

\paragraph{Colonies}




\subsubsection{Insturments of National Power}

\begin{table}
  \begin{center}
  \begin{tabular}{lll}\toprule
    Insturment & British & American \\\midrule
	a & a & a \\\bottomrule
  \end{tabular}
  \end{center}
  \caption{Comparison of British and Colonial insturments of National Power.}
\end{table}

\subsubsection{Military Systems}

\subsubsection{Previous performance}

