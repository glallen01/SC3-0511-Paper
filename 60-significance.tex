\section{Significance of the Action}

\subsection{Immediate Effects on the Campaign}

The Southern Campaign was affected by the British defeat in the Battle of
Cowpens. Lord Cornwallis lost one fourth of his field army in less than an hour.
More importantly he lost his light troops. He would no longer be able to launch
lightning attacks. His favorite cavalry commander, Lieutenant Colonel Tarleton,
could not overcome American Patriots who were mainly comprised of militia and
Continental regulars. The British were greatly demoralized after the loss at
Cowpens. They were astonished with the idea that a united entity of backwoods
militia and Continental regulars quickly and brutally defeated a professional
British army. Americans on the other hand were greatly heartened by this course
of events. Americans felt that they now possessed the initiative. The victory at
Cowpens awakened a hope that the British could be defeated in the Southern
Campaign. 

This battle helped General Nathanael Green build up his army and collect troops.
By dividing his force, he took a great risk.  It was this risk that enabled him
to strengthen his forces and forage throughout the Southern Campaign area of
operation.  After the victory at Cowpens, more and more militia began to join
the Patriots’ cause throughout various parts of country.  Each British defeat
diminished Loyalist support in the Carolinas and increased the influence and
recruitment of the American partisans \cite[p.57]{swisher_duel_2002}.  One outcome that
came from the battle a change in tactics. Morgan proved his mastery of
deployment: he fitted the troops to terrain in different lines of depth, he knew
their strengths and weaknesses and how to best use them
\cite[p.326]{stephenson_patriot_2007}.  Greene used almost the same innovative technique as
Morgan and deployed three defense lines in the battle of Guilford Courthouse.

An operational aspect that appeared in the battle of Cowpens and also later
helped Green at Guilford Courthouse was the effective use of the militia.
Militiamen were not trained to fight in hand-to-hand combat so they were useless
in that kind of warfare, but they were very good marksmen. It was smart place
militia in the front lines, allow them to shoot a few volleys and then retreat
before the advancing British formation could reach them. The other immediate
effect that occurred was to build up an in-depth defense in order to drain the
British as they forced through the battlefield
\cite[p.334]{stephenson_patriot_2007}.  The
course of the war had changed, but there was much to be resolved before it
finally ended \cite[p.138]{babits_devil_2001}. 

After a loss at Cowpens, Cornwallis swore to recover all the British prisoners.
Cornwallis reduced his baggage in order to travel faster and destroy Morgan
before his forces joined with Greene. In mid-March, the two armies fought at
Guilford Courthouse, an action that crippled the British army but left the
American army intact. The British lost the South, and ultimately the
Revolutionary War, largely because of the Continentals, state troops, and
militia from Delaware, Maryland, Virginia, the Carolinas, and Georgia never gave
up. The episode that started the British downslide can be identified as the
Battle of Cowpens, in a large part because the British reaction ultimately led
to their defeat at Yorktown \cite[147-148]{babits_devil_2001}.

\subsection{Long-term effect on the WAR}

After Cowpens Cornwallis did not change his tactics. He was even more furious
after the loss and wanted to find and destroy the Patriot army. He was going
against Sir Henry Clinton`s strategy by leaving South Carolina and entering
North Carolina pursuing Green`s army. Cornwallis was still confident after
retreating to Yorktown but the truth was that Britain had come to the end of its
strategic rope. Cornwallis had nowhere to go.

After the loss in Yorktown, the British were devastated and their empire was
downhearted. Strategically speaking, they had major problems with shipping
resources; they were literally swamped. It was difficult to supply British
troops with basic needs like fuel, fodder and provision. It was difficult to
maintain an army abroad especially over the ocean. Yorktown was the last British
stronghold but by that time it was already too late. The French fleet prevented
relief for Cornwallis and the army besieged in Yorktown was forced to surrender.
The British Secretary of State for the American Department, Lord George Germain,
was ousted  upon the British surrender at Yorktown.  The effective presence and
power of any remaining British entity continued to dwindle and failed to stay
afloat. In March 1782 a new government came into power, dedicated to abandoning
America and ending the war (Rauch, Yale Review).

The decisive factors in the American success were strategic. Britain suffered
its two greatest strategic disasters at Saratoga and Yorktown. The Americans had
better resources to win the war.  They outnumbered the British; they were
skilled in handling weapons and defending a country of limitless space and
resources. The British had colonies all over the world, which made it impossible
to reinforce or even keep up the strength of the force in America. In the end,
the British were forced to withdraw from America (Rauch, Yale Review).

\subsection{Enduring Military Lessons Learned (Principles of War)}

\subsubsection{OBJECTIVE}

\textit{Direct every military operation toward a clearly defined, decisive, and attainable objective (FM 3-0, A-1)}

Brigadier General Daniel Morgan understood what would be expected of him and
his men the day and hours before the Battle of Cowpens.  Morgan knew that he
would be unable to outrun the advancing British forces led by Lieutenant
Colonel Banastre Tarleton.  “The truth probably is that Morgan offered battle
at Cowpens because he could no longer run and had to make a stand.  The British
were hard on his heels, and a swollen river running deep and fast lay across
his line of retreat.  He also was in constant severe pain from his several
ailments—and he was essentially a warrior.  The old Battletown brawler decided
to fight and that was that” \cite[p.125]{lumpkin_savannah_1981}.  The decision to fight and not
flee was in close relation to his orders from Major General Nathaniel Green,
commander of the Southern Continental Army.  Greene provided a clear objective
to Morgan, “…the object of Morgan’s special force was to protect the part of
the country in which it operated, annoy the enemy, ‘spirit up’ the people,
collect provisions, forage out of the way of the enemy, prevent plundering, and
give receipts for whatever was taken, at least to all friends of the American
cause” \cite[p. 120]{lumpkin_savannah_1981}.  The directive allowed Morgan to choose when and
where to fight the British.  Specifically, the guidance from his higher
headquarters provided him the ability to remain at the Cowpens, rest his men,
and pursue the objective to destroy Tarleton’s army and their will to fight.  

General Charles Corwallis expressed the importance of Tarleton’s mission while
at camp in Winnsboro.  “Cornwallis explained to Tarleton his plans for the
coming campaign, and the necessity that Morgan be contained if not destroyed”
\cite[p.306]{buchanan_road_1997}.  Cornwallis’ emphasis on pursuing Morgan stemmed from the
intent of his commander, Sir Henry Clinton.  “His instructions from Sir Henry
Clinton were quite clear: his primary responsibility was the security of
Charleston and South Carolina” \cite[p.307]{buchanan_road_1997}.  Tarleton’s object was to
maintain offensive pressure on Morgan and eventually engage, destroy, and
eliminate the Americans’ will to fight.  Tarleton’s decision to battle with
Morgan at the Cowpens was in line with the strategic objective of the British
Southern Campaign.  The tactical operation, if successfully completed, would
eliminate a threat to Cornwallis’ flank and re-establish Britain’s military
dominance in the South.  The British also hoped that a victory would help
suppress the continued survival of support for a rebellion in South Carolina
\cite[p.307]{buchanan_road_1997}.   

Both Tarleton and Morgan were guided through the same objective that drives
offensive and defensive operations: to destroy the enemy and his will to fight
(FM 3-0, A-1).  On the battlefield, Morgan would achieve the objective through
the use of a defense in-depth.  He would also use the terrain to shadow his
defensive structure to shape a physically and psychologically exhausting war
game for his opponents.  After the British engaged each Continental line of
defense, Morgan’s men would retreat to form a stronger, more destructive line.
This tactic literally destroyed Tarleton’s Army.  

\subsubsection{OFFENSIVE}

\textit{Seize, retain, and exploit the initiative (FM 3-0, A-1)}

Once Morgan made the decision to fight, he took haste to seize the initiative.
He knew the importance of feeding, resting, and allowing his men to relax
before battle.  Morgan did not want his men tense before engaging the British
because he wanted to avoid the common scene of retreating militiamen once the
fight ensued.  Thomas Young described Morgan’s actions the evening before the
battle, “He went among the volunteers, helped them fix their swords, joked with
them about their sweethearts, told them to keep in good spirits, and the day
would be ours” \cite[p.318]{buchanan_road_1997}.   It was important for him to focus his men
on other things and to inspire a desire for them to remain at their lines and
fire their weapons.  

Morgan also took the time to set the stage for his defense by analyzing the
terrain and establishing tactics that would complement his battlefield:

\begin{quote}
“When the command decision was made to fight at Cowpens, Morgan and his staff
rode ahead with local guides and surveyed the planned battle area.  They found
a slope lightly forested in hardwoods and pines, possibly 150 yards long.  This
rose to a low ridge, dipped down to a shallow swale, and rose again to a higher
ridge.  Just behind the crown of the second ridge was a deeper gully in which
cavalry might be concealed.  The depth of the second draw was such that
horsemen could rise in their stirrups and see all the way down the slope to the
forest from which the British must come” \cite[p.126]{lumpkin_savannah_1981}. 
\end{quote}

Morgan would use the slope of the land to conceal the true design of his
defense and tire the British soldiers as they moved uphill to attack the
Americans.  “Morgan’s plan envisioned a defense in depth, consisting of three
linear positions, skirmishers, militia, and Continentals.  Ahead of the battle
lines, he posted pickets, or videttes.  Behind his main-line Continentals, he
placed cavalry as a reserve” \cite[p.72]{babits_devil_2001}.  Morgan would economize his army
and create an evolving defense that would culminate with a main battle line
comprised of war hardened Continental soldiers and militiamen.    The employed
tactics would shape Tarleton’s army into a reckonable force.

Tarleton also understood the importance of exploiting the initiative.  He wrote
General Cornwallis, “When I advance, I must either destroy Morgan’s Corps, or
push it before me over Broad river, towards King’s mountain”
\cite[p.49]{babits_devil_2001}.
When Tarleton arrived at the Cowpens, he assumed that Morgan’s men were in
retreat and that skirmishers had been deployed to delay the British advance.
Tarleton believed that quick action was necessary in order to engage Morgan’s
army before the Continentals were able to cross the Broad River.  Tarleton
moved so quickly that he did not analyze the terrain or allow his entire
formation to emerge from the wood line.  Tarleton understood the importance of
seizing the initiative, but failed gain situational awareness.  Tarleton
rapidly positioned his men in a formation that hindered his ability to fully
employ the strengths of his units.  

“Tarleton ordered the infantry to drop their packs and form into line of
battle.  He did not finish his inspection of Morgan’s dispositions.  He did not
wait until the Highlanders and his main body of cavalry, which he would hold in
reserve, had gotten completely clear of the thick underbrush along Thicketty
Creek.  Nor did he consult with his two veteran infantry commanders, Major
Archibald McArthur of the Highlanders and Major Timothy Newmarsh of the
Fusiliers.  Instead, he formed his main line for an immediate general advance”
\cite[p.321]{buchanan_road_1997}.  

Tarleton’s impulsiveness would prevent his army from ever gaining the
initiative.  His men would be forced to react throughout every phase of the
battle. 

\subsubsection{MASS}

\textit{Concentrate the effects of combat power at the decisive place and time (FM 3-0, A-2)}

Morgan’s defense would incorporate the principle of mass in an effective
manner.  After firing their volleys, each line of his defense would retrograde
into a more powerful, destructive composition.  Morgan took advantage of 18th
century tactics by devising a plan that would engage an enemy advancing in a
rank and column formation.  Morgan massed three lethal lines of defense.  Each
line improved in capability and composition as the British ensued.  The final
line created a massed effect so powerful that it overwhelmed even the most
experienced British units.  Morgan executed a plan that allowed his forces to
engage the enemy at different phase lines, retrograde to the rear to reduce
casualties from a better trained enemy army, and mass at the rear of the
formation.  Morgan also kept his cavalry readily available and allowed
Washington to take advantage of targets of opportunity.  Morgan’s idea to mass
his forces created a situation that allowed his men to immediately support
weakness in his main battle line.  Washington’s location in the rear would also
prove essential to repealing British dragoon attacks and enemy attempts to
penetrate the flanks of Morgan’s formation.   

Tarleton’s decision to quickly dispatch his men and restrictions emplaced by
terrain resulted in his failure to appropriately mass his forces.   His
inability to wait for his entire army to emerge from the woods presented a
problem to the 71st Regiment.  There was not enough room for them to form to
the west of the 7th Regiment while still allowing the Legion Dragoons to
advance on the far western flank, “Tarleton initially desired the 71st to take
position beyond the 7th, but without adequate space to form, the 71st disrupted
the 7th and was then detailed as a reserve.  The Highlanders extended slightly
beyond the 7th’s flank, following about 150 yards behind the line” (Babits,
84).  The lack of establishing the 71st Regiment alongside the 7th was critical
because the 71st was the most experienced of Tarleton’s units.  The Highlanders
were supposed to instill confidence and poise alongside a non-battle tested 7th
Regiment.  Tarleton was also unable to mass the effects of his cavalry,
considered to be his most devastating asset.  American firepower continually
suppressed any advance of the British dragoons and became totally ineffective
once engaged by Washington’s cavalry.  The influence of cannon was also
diminished.  As the British emerged from the brush, they immediately came under
fire from the American skirmishers.  This forced Tarleton to hurry his forward
advance in order to reduce casualties.  This action prevented Tarleton’s
artillery from firing on the massed areas of Morgan’s defense because of risk
of fratricide.   

\subsubsection{ECONOMY OF FORCE}

\textit{Allocate minimum essential combat power to secondary efforts(FM 3-0, A-2)}

Morgan employed his men in a method that allowed him to use the minimal amount
of men necessary to shape his defensive operation.  He used his first line of
skirmishers to harass the British soldiers and force Tarleton to quickly form
and march his men forward.  The skirmishers also effectively repelled an
initial attack and reconnaissance of Legion Dragoons.  Morgan designed his
first two lines of defense to create casualties amongst the British army while
avoiding large American casualties.  The design required a minimal amount of
men, allowing Morgan to allocate the majority of his resources to his main
battle line.  His plan also incorporated each line to build on the other,
eventually creating a main battle line that was composed of men from the
initial two defensive lines.  Morgan also kept the cavalry in reserve providing
Washington with instructions to provide relief wherever he deemed it necessary.  

Tarleton inability to mass his force directly impacted his inability to
economize his force.  The British were engaged from the moment they appeared on
the battlefield.  Tarleton did attempt to initially shape his offensive
operation with the dispatch of cavalry, but the Americans were able to
successfully avoid any considerable impact from this action.  Tarleton avoided
extensive bombardment of the Americans with cannon because he wanted to swiftly
pursue what he believed to be a retreating Continental army and to quiet the
harassment of rifle fire from the skirmishers.  Tarleton also placed the 71st
Regiment in reserve, failing to fully provide purpose to one of his most
decorated units.    

\subsubsection{MANEUVER}

\textit{Place the enemy in a disadvantageous position through the flexible
application of combat power (FM 3-0, A-2)}

Morgan displayed a maneuver masterpiece at the Battle of Cowpens.  The American
forces were implemented in such a manner that maximized the effectiveness of
maneuver by keeping the British forces off balance.  The defense in depth
diminished Tarleton’s overall quantitative strength and allowed the Americans
to preserve freedom of action.  The retrograde of Morgan’s first two defensive
lines forced the British to follow in the exact manner that Morgan desired.
Every line of defense created a new problem set for the enemy.  Tarleton could
not halt his army and fire at the skirmishers or the militia because the
Americans were too quick to flee.  Morgan created a situation that made
Tarleton continue his advance.  Tarleton would pursue in hopes of eventually
arriving at the American main line, which would finally allow his men to fight
in a familiar methodology.   Tarleton would attempt unsuccessfully to flank
both sides of Morgan’s main line with cavalry.  At one point during the battle,
Morgan’s western flank began to crumble as the 71st Regiment pushed through the
main line.  Noticing the need for reinforcement, Morgan reestablished command
of his fleeing militiamen and emplaced them in a manner that halted any British
momentum.  Washington also identified the weakness and maneuvered his cavalry
to the western flank to eliminate the threat of the British dragoons.  

\subsubsection{UNITIY OF COMMAND}

\textit{For every objective, ensure unity of effort under one responsible
commander (FM 3-0, A-3)}

Both the American and British armies were under control of a single commander.
The Americans were led by General Daniel Morgan and the British were under the
command of Lieutenant Colonel Banastre Tarleton.  The British appeared to have
the advantage in organizational leadership because of their training.  Soldiers
trained with their leaders and knew who to look for in battle. Also, a chain of
command existed that defined who would assume control of the unit if an officer
was killed in battle.  In Morgan’s army, he had a more difficult situation.
Although Morgan did have Continental Regulars in his formation, he also had a
conglomerate of militias from across the South.  This could have been cause for
concern if a militia’s loyalty resided with another commander not willing to
take direction form General Morgan.  Fortunately for the Americans, Morgan was
a well-respected leader with proven competence.  The militiamen were eager to
follow his orders.     

\subsubsection{SECURITY}

\textit{Never permit the enemy to acquire an unexpected advantage (FM 3-0,
A-3)}

Morgan used the terrain and his cavalry to strengthen his defensive plan.  The
marshland on the edges of the battlefield would prevent Tarleton from extending
his formation in a decisive manner.  Instead, Tarleton would be forced to place
his most capable unit in reserve simply because it did not fit on his main
battle line.  Morgan would extend his main battle line in such a way that it
would force any flanking attempt by Tarleton’s cavalry to approach from
inhibiting terrain.  Any other avenue of approach for a flanking movement would
place British forces in front of American muskets.  

Morgan used surveillance and reconnaissance to ensure operational security.
His scouts were able to correctly track Tarleton’s movements and provide an
accurate timeline for a possible rendezvous.  Tarleton also had scouts that
provided informational reports on Morgan’s whereabouts, but failed to provide
any valuable operational feedback.  Tarleton did not know that Morgan was
preparing for battle and assumed that the Continentals were in full retreat
upon his arrival at the Cowpens.  Lack of valuable British intelligence
contributed to Tarleton’s failure and allowed Morgan to fully employ deception
on the battlefield.   

\subsubsection{SURPRISE}

\textit{Strike the enemy at a time or place or in a manner for which he is
unprepared (FM 3-0, A-3)}

Tarleton’s inability to obtain significant intelligence preparation of the
battlefield (IPB) allowed Morgan to fully exploit a plan of surprise.   “Now,
on the battlefield he had chosen, Morgan laid a tactical trap, taking advantage
of Tarleton’s aggressiveness and perceptions of how a battle was fault”
\cite[p.61]{babits_devil_2001}.  General Morgan intellectually wagered that Tarleton would
employ the British army in an overconfident manner.  Tarleton assumed his force
would easily achieve victory based on previous awful performances of the
Continental and militia armies.  “Morgan’s trap depended on breaking down the
British, and, if Tarleton could not see the American lines, the later
appearance of new, stronger lines would come as something of a surprise”
\cite[p.82]{babits_devil_2001}.  Morgan used the terrain to shield his established defense from
enemy observation.  This effect created a physical and physiological impact on
Tarleton’s army as they advanced from one of Morgan’s defensive lines to the
next.  Morgan would climax the element of surprise by emplacing his strongest,
most experienced battle ridden men as the main line.  The Americans would
unleash fury on the weary, depleted British army.   

\subsubsection{SIMPLICITY}

\textit{Prepare clear, uncomplicated plans and clear, concise orders to ensure
through understanding (FM 3-0, A-3)}

General Morgan provided clear and concise guidance to his commanders on the
execution of his defensive plan.  “Morgan’s instructions were first issued to
unit commanders on the scene if they had to fight.  Later, as other officers
arrived, they, too were briefed about their roles but apparently given only
partial views of the whole plan” \cite[p.54]{babits_devil_2001}.  Morgan understood the
importance of informing the battlefield leaders of what he expected to
transpire.  Ensuring that his leaders knew their individual tasks would prevent
confusion on the battlefield and allow his subordinates to execute their tasks
without continual supervision from the battlefield commander.  It would be
impossible for Morgan to execute his defense in depth without competent
subordinates who understood their individual mission.  Morgan could not be
physically present at every operational phase line.  He not only attempted to
eliminate the possibility of confusion amongst his unit commanders, but also
from his all officers that would be present on the battlefield.  Morgan
instructed them on their individual roles to prevent misunderstanding.  

Lieutenant Colonel Tarleton did not devise a simple offensive plan prior to
arriving at the Cowpens.  He was at a disadvantage due to his belief that he
was about to engage a fleeing enemy and not Morgan’s well planned positioned
defense.  Tarleton’s commands on the battlefield were simplistic in nature, but
his overarching concept of the offensive engagement was not briefed to his
infantry commanders \cite[p.321]{buchanan_road_1997}.  Amongst the chaos of battle as the
skirmishers fired upon the assembling British line, confusion existed between
units.  The 71st Regiment collided with the 7th Regiment as they tried to
fulfill Tarleton’s guidance to position themselves on the left flank of the 7th
Regiment.  Tarleton initially failed to realize his available battle space,
issued orders that were executed correctly, and had to reposition his 71st
Regiment into a rear reserve.   The confusion amongst the British army would
continue throughout the Battle of Cowpens as Tarleton’s army engaged unfamiliar
enemy tactics.  

\subsection{Insights for Contemporary Military Professionals}

Analyzing the Battle of Cowpens generates many applicable insights learned that
are relevant to the military professional today.  The Battle of Cowpens
illustrates the fact that although an army may be a considerably weaker
(Continentals) threat than its opposing force (British), the weaker army is
still capable of destroying a formidable enemy.  An army can fail even if it
possess the advantage in training, equipment, and a more lethal capability of
maximizing the effects of combined arms.  The Battle of Cowpens serves as a
reminder to all military leaders that it is necessary to remain humble when
pursuing an enemy that may be deemed as inferior.    A leader must evaluate
every enemy as a greater or equal force.  The importance of collecting,
analyzing, and distributing intelligence can prove to be the difference in the
outcome of an engagement.  A well informed commander is capable of making
better decisions when he knows information regarding his enemy and the elements
that surround the place of battle.  A leader must always strive to limit the
impact of the fog of war.  As a situation develops, he must continually
consider the impact that his orders will have on his soldiers and the mission.
A leader must also take the time to understand the environment.  One must know
the weather, recognize the effects that terrain may have on equipment and
Soldiers, care for the troops, clarify commander’s intent to subordinates, and
establish a plan before engaging in offensive or defensive operations.

There are many applications to current military operations.  Military leaders
must force themselves to become creative problem solvers.  The United States
continues to face an adaptive enemy.  When tactics, techniques, and procedures
(TTPs) are understood by the enemy, we become most vulnerable to harm.  We must
understand the importance of differentiating our TTPs on a frequent enough
basis in order to prevent the enemy from knowing how our military will react to
a specific situation.  This can be something as simple as distance between
vehicles while on patrol in Afghanistan or the way dismounted Soldiers are
taught to react to sniper fire.  We cannot wait for the Army to dictate new
ideas because it takes too long.  Leaders at all echelons must remain vigilant
of how operations are conducted by their units.  We must force our Soldiers to
develop various battle drills for common military operations and instill in
them a true understanding of the dangers of complacency.  

Current operations also dictate the need for command and control at the lowest
tier.  Commanders must establish clear and concise guidance in regards to all
missions and operations.  Subordinates must understand what is expected of each
of them before they execute.  An informed Soldier is an effective Soldier.  It
is extremely common for junior ranking Soldiers to lack an understanding of
their mission requirements.  They must know the essential impact that their
actions have on the success of their team, squad, platoon, or company.  Today’s
wars are fought at the lowest echelon.  Before a platoon or company departs on
a real world mission, it is necessary to inform every individual of his purpose
and the overall required end state.   This allows for Soldiers, Noncommissioned
Officers, and Officers to continually evaluate the effects of their decisions
throughout a mission.  Once an individual knows why he is doing something,
every action has greater meaning.  If leadership is lost during an engagement,
the remaining element can still complete the desired end state.  

Our military is deployed across two countries half way around the world.
Relations with neighboring countries, such as Kuwait and Pakistan, are
essential to maintaining key logistic routes.  During the American Revolution,
the Americans were able to win the support of Spain and France allowing them to
purchase weapons and supplies to sustain a war against Britain.  Pakistan and
Iran are believed to support the Taliban and Al Qaeda factions, completely
undermining the United States’ mission in Afghanistan.  There are claims that
Taliban and Al Qaeda elements have been able to receive weapons and explosives
from the neighboring countries causing increased US casualties over the past
couple of years.  The significance of this is attrition.  In the Revolutionary
War, the American rebellion was able to survive because of support from
wealthier, outlying countries.  Spain and France were undermining any
successful attempt of the British to win the war.  This allowed the Americans
to successfully partake in guerilla type tactics and harass the British Army to
the extent of inflicting significant damage to soldiers and logistics.
Eventually, this lead to English sentiment significantly declining because the
war had gone on far longer than people felt it should.   America finds itself
in a similar situation today.  US efforts are continually undermined by the
availability of support from anti-US nations and their people.  The war in
Afghanistan recently became the longest war in US history.  A great percentage
of American society is fed up with the lack of results from spending billions
upon billions of dollars on our Global War on Terrorism.  Public dissent has
caused a decline in support for the war from the House and Senate and an
increased demand from politicians to request that our military forces be
brought home.  Military leaders deem that success in Iraq and Afghanistan rely
on winning the hearts and minds of their populations, but we often fail to
realize that we are most subject to the desires of our own society in America.     


