\section{The Strategic Setting}

\subsection{Causes of the Conflict}

There are many factors of politics, trade, economics and oscity that all
affected the colonies break away from Britain One analysis of the American
Revolution sees it as a continuation of the Seven Years War. Another focuses on
its nature as a revolt against tightening of mercantilism around the colonies,
who were slowly gravitating towards idas of trade codified by Adam Smith's
``wealth of Nations,'' published in \red{1776}. The popular analysis lies in a
revolt against laws seen as unjust, levied without representatoin. This section
will address in turn the previous war, mercantilism, the influence of other
foreign powers, and finally the series of political events that drew a line in
the sand for the colonies.

%%% --begin-notes
\subsection{--notes--}\footcite[2]{moncure_cowpens_1996}

Enlightenment political theory:

French and Indian War (1754-63)

Village green in Lexington, MA 19 APR 1775

trade unrestricted prior to 1760s

King sends:\\
Major General Sir William Howe\\
Major General Sir Henry Clinton\\
Major General Sir John Burgoyne

\paragraph{--notes--}\footcite[]{}
``Shy. John, \emph{A People Numerous and Armed.} pp. 198-199 and 230-232. The British contemplated cutting loose New England and retaining the richer lands of the South in 1778. but never pursued a negot~ated settlement with the Continental Congress.''

%%% --end-notes

\begin{quotation}
  What do we mean by the Revolution? The war? That was no part of the Revolution;
  it was only an effect and consequence of it. The Revolution was in the minds of
  the people, and this was effected, from 1760 to 1775, in the course of
  fifteen years before a drop of blood was shed at Lexington. The records of
  thriteen legislatures, the pamphlets, newspapers in all the colonies, ought
  to be consulted during that period to ascertain the steps by which the public
  opinion was enlightened and informed concerning the authority of Parliament
  over the colonies.\footnote{John Adams to Jefferson, 1815}
\end{quotation}

\subsection{Mercantilism}

\subsection{--notes--: Cook}

King George III ascended to the throne in 1760. He succeeded two German
speaking kings, and was the first native-born soverign over England in
\red{???} years. Along with these other differences, he was at odds with the
Whig party which has previously driven politics and policy in England for
\red{??? years}. His transition was three years prior to the end of the Seven
years war, which ran from \red{???} to 1763. And, although crowned following a
series of victores, George III's first proclamation as ruler reflected his
desire to end the war, rebuild a war-burdened economy and country that had
already reached it's highest level of debit so far.


