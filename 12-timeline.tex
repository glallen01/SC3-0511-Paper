\begin{singlespace}
\ctable[
  cap={Events leading to the Revolution},
  caption={Timeline of events leading to the American Revolution.},
  pos=h,
  ]{>{\footnotesize}r>{\footnotesize}p{4in}}
  {
    \tnote{ Sources include: \cite[323-326]{berard_1878}, \cite{cook_long_1995}.
    }
  }{\FL
Year 		& Event 									\ML
%~		& \red{Currency Act:}								\NN
%~		& \red{Quartering Act:}								\NN
%~		& \red{Intolerable Acts:}							\NN
%~		& \red{End of Sir Robert Walpole's `Salutary Neglect' policy (Cook, 3)}		\NN
%~		& \red{Boston Tea Party}							\NN
%1765		& \red{Stamp Act}								\NN
%Apr. 1764	& \red{Sugar Act:} passed prompting cries of ``Taxation without Representation''\NN
%Jan. 1775	& Benjamin Franklin departs London a `determined revolutionary' (Cook, xii) 	\NN
%1776		& \red{Adam Smith publishes the \emph{Wealth of Nations}}			\NN
%Sep. 1783	& The American Revolution ends with the treaty of !!!				\NN
%~ 		& ~ 										\LL
%
%
%1749 		& Commandant C\'eloron lays claim to the Ohio Valley for France 		\NN
%\red{Nov. 1749}	& ?? Navigation Acts								\NN
1754		& French begin the French and Indian War in the states, which
		  spreads to Europe to become the Seven years War.\NN
%Jan. 1757	& Benjamin Franklin travels to London as a Colonial representative		\NN
%Oct. 1760	& King George III ascends to the throne of England				\NN
1761		& Writs of Assistance\NN
%Nov. 1761	& James Otis challenges the Writs of Assistance, allowing searches to prevent
%		  smuggling, as violations of the constitutional rights of colonists.		\NN
1763		& Treaty of Paris ends the Seven Years' War\NN
Oct. 1763	& Proclamation of 1763: King George III forbids colonists settling west of
		  the Appalachian Mountains.							\NN
Mar. 1765	& Stamp Act\NN
Oct. 1765	& Colonial Congress\NN
Mar. 1766	& Stamp Act Repealed\NN
Oct 1768	& Troops sent to Boston\NN
1770		& Boston Massacre\NN
16 Dec. 1773	& Boston Tea Party\NN
Sept. 1774	& Continental Congress\NN
19 Apr. 1775	& Battles of Lexington and Concord\NN
4 July 1776	& Declaration of Independence\NN
1777		& \red{Washington at Valley Forge}\NN
1778		& \red{French Treaty.}\NN
29 Dec. 1778	& Savannah captured.\NN
12 May. 1780	& Loss of Charleston\NN
17 Jan. 1781	& Battle of Cowpens\NN
15 Mar. 1781	& Battle of Guilford Courthouse\NN
19 Oct. 1781	& Cornwallis's Surrender\NN
19 Apr. 1783	& Cessation of Hostilities\NN
3 Sept. 1783	& Treaty of Paris\LL
}
\end{singlespace}

%---------------------
% % http://militaryhistory.about.com/od/americanrevolution/a/amrevcauses.htm
% Rise of Liberalism and Republicanism
% 
% As tensions regarding colonial lands and taxation increased during the 1760s
% and 1770s, many American leaders were influenced by the liberal and republican
% ideals espoused by Enlightenment writers such as John Locke. Key among Locke's
% theories was that of the ``social contract'' which stated that legitimate state
% authority must be derived from the consent of the governed. Also, that should
% the government abuse the rights of the governed, it was the natural
% responsibility of the people to rise up and overthrow their leaders. The ideas
% of Locke and other similar writers contributed to the American embrace of
% ``republican'' ideology in the years before the Revolution. Standing in
% opposition to tyrants, republicanism called for the protection liberty through
% the rule of law and civic virtue.
%--------------------------

