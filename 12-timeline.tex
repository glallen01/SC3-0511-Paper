\documentclass[10pt,letterpaper]{article}

%\usepackage{a4wide}
\usepackage[]{color}
\usepackage{timeline/timeline}
\usepackage{ctable}
\usepackage[margin=1.0in]{geometry}

\pagestyle{empty}

% \title{History of Voting Technology in the United States}
% \author{Tim Storer}

\newcommand{\red}[1]{\textcolor{red}{#1}}
\renewcommand{\eventdate}[2]{
\if 0#1
 \tiny\bf#2
\else
 \tiny\bf\ToShortMonth{#1}, #2 
\fi
}

\begin{document}
%\maketitle

%\definecolor{BGColor}{rgb}{1.0,1.0,1.0}
%\definecolor{ChartColor}{rgb}{1.0,1.0,1.0}
%\begin{timeline}{1745}{1800}{400}{400}
%\YearEvent{1749}{Commandant C\'eloron lays claim to the Ohio Valley for France}
%\MonthAndYearEvent{11}{1749}{?? Navigation Acts}
%\YearEvent{1754}{French and Indian War begins}
%\YearEvent{1754}{?? Seven Years War begins}
%\MonthAndYearEvent{1}{1757}{Benjamin Franklin travels to London as a Colonial representative}
%\MonthAndYearEvent{10}{1760}{King George III ascends to the throne of England}
%\MonthAndYearEvent{11}{1761}{James Otis challenges the Writs of Assistance, allowing searches to prevent smuggling, as violations of the constitutional rights of colonists.}
%\MonthAndYearEvent{2}{1763}{Seven Years War ends with the treaty of !!!}
%\MonthAndYearEvent{10}{1763}{Proclamation of 1763: King George III forbids colonists settling west of the Appalachian Mountains.}
%
%\MonthAndYearEvent{1}{1764}{?? Currency Act: }
%\MonthAndYearEvent{1}{1764}{?? Quartering Act: }
%\MonthAndYearEvent{1}{1764}{?? Currency Act: }
%\MonthAndYearEvent{4}{1764}{Sugar Act passed prompting cries of ``Taxation without Representation''}
%
%\MonthAndYearEvent{1}{1775}{Benjamin Franklin departs London a `determined revolutionary' (Cook, xii)}
%\YearEvent{1776}{\red{Adam Smith publishes the \em{Wealth of Nations}}}
%\MonthAndYearEvent{9}{1783}{The American Revolution ends with the treaty of !!!}
%\end{timeline}

\ctable[
  caption={Timeline of events leading to the American Revolution.}
  ]{lp{4in}}{}{											\FL
Year 		& Event 									\ML
1749 		& Commandant C\'eloron lays claim to the Ohio Valley for France 		\NN
\red{Nov. 1749}	& ?? Navigation Acts								\NN
1754		& French and Indian War begins							\NN
\red{1754}	& ?? Seven Years War begins 							\NN
Jan. 1757	& Benjamin Franklin travels to London as a Colonial representative		\NN
Oct. 1760	& King George III ascends to the throne of England				\NN
Nov. 1761	& James Otis challenges the Writs of Assistance, allowing searches to prevent
		  smuggling, as violations of the constitutional rights of colonists.		\NN
Feb. 1763	& Seven Years War ends with the treaty of !!!					\NN
Oct. 1763	& Proclamation of 1763: King George III forbids colonists settling west of
		  the Appalachian Mountains.							\NN
~		& \red{Currency Act:}								\NN
~		& \red{Quartering Act:}								\NN
Apr. 1764	& \red{Sugar Act:} passed prompting cries of ``Taxation without Representation''\NN
Jan. 1775	& Benjamin Franklin departs London a `determined revolutionary' (Cook, xii) 	\NN
1776		& \red{Adam Smith publishes the \emph{Wealth of Nations}}			\NN
Sep. 1783	& The American Revolution ends with the treaty of !!!				\NN
~ 		& ~ 										\LL
}

%---------------------
% % http://militaryhistory.about.com/od/americanrevolution/a/amrevcauses.htm
% Rise of Liberalism and Republicanism
% 
% As tensions regarding colonial lands and taxation increased during the 1760s
% and 1770s, many American leaders were influenced by the liberal and republican
% ideals espoused by Enlightenment writers such as John Locke. Key among Locke's
% theories was that of the ``social contract'' which stated that legitimate state
% authority must be derived from the consent of the governed. Also, that should
% the government abuse the rights of the governed, it was the natural
% responsibility of the people to rise up and overthrow their leaders. The ideas
% of Locke and other similar writers contributed to the American embrace of
% ``republican'' ideology in the years before the Revolution. Standing in
% opposition to tyrants, republicanism called for the protection liberty through
% the rule of law and civic virtue.
%--------------------------

% \MonthAndYearEvent{1}{1749}{Intolerable Acts}
% \MonthAndYearEvent{1}{1749}{End of Sir Robert Walpole's `Salutary Neglect' policy (Cook, 3)}
% \MonthAndYearEvent{1}{1749}{Boston Tea Party}
% \MonthAndYearEvent{1}{1749}{Stamp Act}
% \MonthAndYearEvent{1}{1749}{}
% \MonthAndYearEvent{1}{1749}{}



\end{document}
