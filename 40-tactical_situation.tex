\section{The Tactical Situation}


\subsection{The Area of Operations}

%\subsubsection{Climate and Weather}

Due to the nature of the battle reports and personal anecdotes of the
participants, there is not a great deal of information regarding the climatology
and weather conditions of the Battle of Cowpens.  However, a historical analysis
by the South Carolina State Climatology Office indicates a January average
minimum temperature of 25$^\circ$ Fahrenheit and average maximum of 50$^\circ$ Fahrenheit from
1971 to 2000 for the area surrounding the battle (SCSCO).  For the purposes of
this paper it will be assumed that this trend also applies for the month of
January 1781.

\begin{quote}
``It was a bitter cold morning, and the soldiers slapped their hands to keep warm
as they waited in the dark for the British troops to arrive.  No evidence
remains of the time the engagement began; not even the sun would signal the
onslaught on this overcast day'' \cite[51]{moncure_cowpens_1996}.  This statement, along with
others, indicate that it was a very cold, overcast day; this low  temperature
and high humidity would have affected the combatants by stiffening their fingers
and joints, slowing marching speeds, and hindering fine muscle movements, such
as reloading their weapons and fixing bayonets. Soldiers were not alone in being
impacted by the cold and damp, their equipment also suffered.  ``Dampness made it
difficult for flintlocks to fire or to ignite rapidly, affecting accuracy'' \cite[79]{babits_devil_2001}.
It would also have had a negative outcome upon morale. 
Visibility on both sides was limited: ``Low clouds or mist affected any
assessment of troop dispositions, even after full daylight.  Combined with
ground cover and elevation, mist may have blocked Tarleton's ability to see the
Continentals waiting on the main line'' \cite[80]{babits_devil_2001}.
\end{quote}

The climate on the morning of the battle appears to have benefited the
Continentals in that, even though both forces had to face the hardships of
weather, the Americans had been able to rest, unlike Tarleton's forces who had
to march all night to meet them at Cowpens.  Tarleton’s men were likely feeling
the greater effects of the cold.  It also mitigated another concern for Morgan:
his men would have been looking directly into the sun during the battle, had the
day not been overcast: “Clouds benefited the Americans, who would otherwise have
had the early morning sun in their eyes.” \cite[67]{moncure_cowpens_1996}.  

\subsubsection{Terrain}

The landscape at Cowpens, given current advances in tactics and technology,
would be considered a ``killing ground'' or optimal ambush point.  Considering the
equipment and combat tactics of the time, it was a perfectly suited battleground
for the forces involved.  Tarleton considered it ``An open meadow (for cattle
grazing) about 500 yards deep and the same wide, gently undulating, not much in
the way of trees and undergrowth; good terrain to maneuver infantry; good space
to use cavalry once the enemy was broken.  The Mill Gap road bisected the field
North to South'' \cite[326]{stephenson_patriot_2007}.
From Morgan's perspective the landscape was ``relatively flat open ground
sparsely scattered with red oak and pine, the site was ideal for grazing cows or
fighting European style battles.  From the direction the British must come
(Northwest along Mill Gap Road), a single trail opened into a narrow plain that
sloped gently but unevenly uphill to the center of the pens'' \cite[45]{moncure_cowpens_1996}.
The only avenue of approach for the British was the Mill Gap Road, a trail that
had become extremely muddy due to recent rains in the region. ``The advance of
the British in the dark across a country cut by ravines and swollen, muddy
creeks was slow'' \cite[126]{lumpkin_savannah_1981}.  Figure \ref{terrain1a}
shows the terrain surrounding the battle site.  

\begin{figure}[h]
    \begin{center}
    \includegraphics[width=\textwidth]{gfx/futch1}
    \end{center}
    %\caption{Terrain overview of the Battle of Cowpens. \cite{wilson_blogmap}}
    \caption{Phase 1: British approach at Cowpens. \cite{wilson_blogmap}}
	%\url{http://www.davidwilsonhome.com/homepage/Perspectives/Entries/2010/4/11_The_Battle_of_Cowpens__A_Cartographic_Interpretation.html)}
    \label{terrain1a}
\end{figure}

Morgan, in seizing what limited high ground there was, had the advantage of
Tarleton in regards to observation during the British infiltration, but any
benefit was negated during the battle due to the gently sloping terrain once
Tarleton’s forces had entered the meadow.  The British’s cannons were not
influenced by the terrain at all. 

For a relatively small battlefield, there was a great deal of important or key
terrain to be had.  One of the primary ones was a series of crests near the
center of the field: “They found a slope lightly forested in hardwoods and pine,
possibly 150 yards long. This rose to a low ridge, dipped down to a shallow
swale, and rose again to a higher ridge.  Just behind the crown of the second
ridge was a deeper gully in which cavalry might be concealed.  The depth of the
second draw was such that horsemen could rise in their stirrups and see all the
way down the slope to the forest from which the British must come”
\cite[126]{lumpkin_savannah_1981} \cite[126]{lumpkin_savannah_1981}.
This saddle in between two crests was the perfect depth to both shield
cavalry from attacking forces and to allow them visibility of the battle with
little effort. Another piece of key terrain was the marshland that braced both
sides (east and west) of the Continental line.  ``The flanks were close to the
marshy ground of two creeks that bracketed the Cowpens'' \cite[327]{stephenson_patriot_2007}.
The marsh on both sides of the battlefield created a canalization effect for the
attacking British, limiting their ability to flank the Continental forces.  

Operationally, the terrain had a different impact for each side.  Morgan used
the gully to obscure his cavalry and the marshes on either side of his forces to
limit the possibility of an enemy flank.  Tarleton appears to have disregarded
those factors and did not account for their impact during the battle.  “Morgan
noted the way the land fell off to the left and right toward several creeks.
The Cowpens was bordered by marshy ground that would make it difficult for
Tarleton to execute any sweeping flank movements with his cavalry” \cite[45]{fleming_cowpens_1988}.

The aforementioned marshes were a key obstacle in the fight, hemming and
canalizing the British into fighting straight on and limiting any flanking
movements.  That being said, the area in which the majority of the fighting took
place in was ``an open rolling woodland of first-growth pines and hardwoods,
excellent country for cavalrymen but with very little cover for riflemen''
\cite[124]{lumpkin_savannah_1981}.  The terrain definitely influenced the battle, it allowed
Morgan to anchor his forces to the low hill in the center and tie them into the
edge of a marsh on the eastern side of the battleground, and it led Tarleton to
make an assumption that this fight was going to be fought the way he was
accustomed to: ``the ground was optimal for Tarleton's cavalry, as the British
commander noted, although it sloped gently upward toward the American forces''
\cite[46]{moncure_cowpens_1996}.

Given the nature of the terrain and the style of warfare at the time, there was
not much in the way of cover or concealment to be had.  In fact, ``all sources
agree the battlefield was partially open with `not one single bush on the field
of battle to entangle the troops.'''\cite[66]{babits_devil_2001}.  Morgan made
excellent use of the dip in elevation behind his troops: ``Behind the main line,
in a shallow gully, Morgan parked William Washington's 3rd Continental Light
Dragoons (82 men) together with 45 volunteer horsemen drawn from the militia''
\cite[327]{stephenson_patriot_2007}.  The lack of cover or concealment clearly
led Tarleton to believe that this fight would be on his terms and fought in a
manner in which he was familiar.  


%\paragraph{Observation and Fields of Fire}
%
%\ldots
%
%\paragraph{Avenues of Approach}
%
%\ldots
%
%\paragraph{Key Terrain}
%
%\ldots
%
%\paragraph{Obstacles}
%
%\ldots
%
%\paragraph{Cover and Concealment}
%
%\ldots
%
%\subsection{Comparison of Opposing Forces}
%
%\ldots

\subsubsection{Strength and Composition}


At the Battle of Cowpens, the British had approximately 1000 - 1200 troops.
Tarleton’s force consisted of “550 dragoons and light infantry of the British
Legion, the 1st Battalion of the 71st Foot (200 men), a similar number of the
7th Foot, 50 horsemen of the 17th Dragoons, and a 50-man contingent of the Royal
Artillery with two three-pound grasshoppers; in all, just over 1,000 men”
\cite[p.325]{stephenson_patriot_2007}.  The 7th Foot was commanded by Major Timothy Newmarsh.  It
was comprised of about 168 men arranged in four companies.  The 71st Foot was
known as Fraser’s Highlanders and was raised specifically for American service
in 1775.  At the Battle of Cowpens the 71st Foot was comprised of 249 men and 14
officers.  British companies had approximately forty men in each company.  The
British Legion Infantry was comprised of 200 – 271 enlisted Infantry men and
nearly 250 horsemen. It was normally commanded by MAJ George Hanger however he
was absent due to sickness at the time of Cowpens, his most likely replacement
was Captain John Rousselet.  The 17th light Dragoons consisted of 50 horsemen
commanded by Lieutenant Henry Nettles.        

The   Continental Army outnumbered the British army by just a little bit.  It is
estimated they had between 1,100 – 1,300 troops under the command of Morgan.
Even though the Continental Army outnumbered the British, they were not as well
equipped, trained, or experienced giving the British the advantage as to troops.
The most experience troops Morgan had were the 300 man Battalion of Continental
troops received from General Greene.  The battalion of consisted of 5 companies
and each company was comprised of 60 men.  The battalion of continental troops
was commanded by LTC John Eager Howard of Baltimore.  Daniel Morgan received the
cream of the crop of these troops.  The rest of his army consisted of 82 light
dragoon riders that were greatly under equipped and under strength along with
militia forces that came scattered throughout the area.  “In addition to
Continental light infantry, a Virginia militia battalion under Major Francis
Triplett saw extensive service with the flying army.  North Carolina militiamen
under Colonel Joseph McDowell operated a second militia battalion under Morgan.
Finally, cavalry composed largely of the remnants of the Third Continental
Dragoons under William Washington completed the components of the Flying Army”
\cite[p.25]{babits_devil_2001}.

\subsubsection{Weapsons Technology}


The types of weapons both the British and Continental Armies used in the
revolutionary war played critically as to how battles were fought and won.  Each
Army had different advantages over the other in types of weapons they used.  The
British had standard weapons throughout its troops while the Continental had a
vast variety of weapons. The weapons comprised at the Battle of Cowpens
included, muskets, rifles, sabers, carbines, bayonets and artillery cannons.  

The Continental Infantrymen mainly carried a French Smoothbore musket with a
bayonet attached.  These French muskets were loaded with a .69 caliber round
with was .63 inches in diameter.  A smoothbore musket has no rifling and does
not allow the projectile to spin extremely diminishing the accuracy of the
round.  In 1777, General George Washington ordered the use of buckshot to be
used, meaning buckshot was used at the Battle of Cowpens.  “Buckshot consisted
of one large ball and at least three small (.30 Caliber) balls.  This allowed
one single Company of 60 men to launch 240 projectiles in 1 single volley”
\cite[p.12]{babits_devil_2001}.  The technique greatly enhanced the amount of firepower of one
company allowing four times as much damage per volley.

The British Infantrymen also carried muskets referred to as the Brown Bess.
These weapons were British made and used a .75 caliber round and had a 17 inch 3
sided bayonet attached.  The three sided bayonet would cause more damage than a
bladed bayonet by creating a larger hole in their opponents’ body.  This made it
much harder for anyone to close the wound allowing it to heal.  The British
Officers and Non Commissioned Officers carried swords with them.  Many British
Officers would also carry spontoons.

The Continental Cavalrymen were not equipped as well as the British Cavalrymen.
The primary weapon of the Cavalrymen was their sabers.  “The militia obtained
sabers by going to all the sawmills.  They would take all the old whip saws and
set three of four smiths to work to create a Saber.  Most swords were polished
with Grindstone” \cite[p.21]{babits_devil_2001}.  This reference shows that the Continental Army
had to be very resourceful when it came to sabers.  Cavalrymen would also carry
a .67 caliber French Carbine.  The handguns were nine inches in barrel length.
It is estimated that for every four Continental Cavalrymen only one would have a
carbine.

The British Cavalry, also referred to as dragoons, were much better equipped
than the Continentals.  Each rider carried a saber that was long, heavy and had
a deeply curved blade like a crescent.  Each Saber had a single edge and a
knuckle guard to protect the rider’s hand.  Each British Cavalrymen also carried
a short musket or a carbine.  These carbines carried a .62 or .65 caliber round.
Even though each rider had a carbine, these weapons were not their primary
weapon and did not do near as much damage as a dragoon charge.

The Continental forces main weapon advantage over the British was the rifle.
The rifle was far more accurate than the musket for it would induce spin on the
projectile.  “Rifles used at Cowpens fit a generalized pattern with a barrel
length usually over 40 inches, a bore averaging .40 to .60 caliber (with seven
or eight grooves); a long thin stock extended to the muzzle, and a patch box.
The rifles weighed six pounds, give or take a few ounces, with a ball as small
as thirty-six to the pound, or about .50 caliber.  Rifleman could hit a target
between 200 – 300 yards away; much further than a musket.  American rifles used
about as much powder as is contained in a woman’s thimble”
\cite[pp.13-14]{babits_devil_2001}.
The rifles had a rear sight allowing the rifleman to take aim.  These rifles
were individual personal weapons and varied in bore sizes.  However, there were
many drawbacks to this weapon.  Logistically it was a nightmare to provide
ammunition.  Consequently each rifleman had to make their own ammunition.  “The
range of bores created problems for supply officers, consequently they issued
riflemen lead bars to make bullets, using molds made for their individual
weapon.  One Pound lead bars were provided to riflemen marching through
Salisbury during the Cowpens campaign” \cite[p.14]{babits_devil_2001}.  Each rifle also had a
much slower reload time than a musket.  A very good rifleman could shoot
approximately one round every 15 seconds.  Rifles were also unable to carry a
bayonet.  Because of this, rifleman companies were a very weak counter against a
charge from enemy troops.

The British had two artillery cannon at the Battle of Cowpens.  The British
artillery consisted of two three-pound guns, meaning each could fire a single
shot round that weighed three pounds.   These guns were also called grasshoppers
because of the style of their carriage.  These cannon could also fire grapeshot
which was used to breakup enemy troops at a closer distance.  The preferred
tactic used by artillery fire was call ricochet firing. “British artillerist
John Muller recommended ricochet firing because it saved powder and was more
dangerous.  After the first ricochet, a ball might bounce another 400 yards and
still injure men waiting in reserve.  Shot was bounced into enemy ranks, because
it caused the disorder necessary before ordering a bayonet charge.  Ricochet
firing might also create panic because the enemy could see the shot coming”
\cite[p.21]{babits_devil_2001}.  Ricochet firing could injure men up to 400 yards back in
reserve.  Artillery cannon were normally placed in the flank position so it
could fire across the attacker’s front.  These three pound cannons were meant to
be moved alongside with the Infantry for direct support.  Even though the
British has this artillery asset, it did not make a significant impact on
Morgan’s plan to defeat the British in battle.   

\subsubsection{Sustainment and Logistics}


Both armies had great difficulties with their supply and transport problems
throughout the war.  Both could not develop a system that worked efficiently.
For the Americans, Congress had many difficulties making decisions on how to
resupply the Continental Army.  It was not until October 1777 when members were
added to Congress that possessed experience in the Quartermaster and Commissary
departments.  The original plan of the British was to “live off the land,
sustaining itself with food it would find in America”
\cite[p.115]{stephenson_patriot_2007}.  That
however, did not work out; Britain had to ship its supplies from across the
Atlantic.  This was a nightmare, the length in travel took too long for
sufficient supplies to arrive and the British did not have enough ships to
deliver the amount of supplies in demand.  The ships were described as
“banged-up old tubs, and to travel in them for six to eight weeks of an average
transatlantic voyage was to experience a circle of hell that would have turned
Dante green” \cite[p.118]{stephenson_patriot_2007}.     

On both sides the quartermaster general and commissary general were essential in
sustaining the war.  “The quartermaster general, working closely with the
commanding general, helped plan marches, select campsites, survey roads and
bridges, and, where necessary repair them.  He was responsible for all
transportation for the army on the march and for providing it with shelter in
camp” \cite[p.103]{stephenson_patriot_2007}.  The Commissary General’s function was described as
follows,  “The Commissary General’s primary role was to feed the army.  Like the
quartermaster general’s job, it was something of a poisoned chalice.  The
responsibility was massive, the resources provided by congress were inadequate,
and the situation was compounded by frequent reorganizations and reforms,
sometimes at the most inappropriate times, such as in the middle of campaign
season” \cite[p.104]{stephenson_patriot_2007}.  

Transportation was insufficient for both sides, often bogged down by bad weather
and awful road conditions.  “Transportation was the linchpin of supply and
throughout the war was a logistical headache for the quartermaster general’s
departments of both armies.  The road system was sparse, the terrain rugged.  In
bad weather the roads turned to quagmire.  It took over two days to travel 90
miles between New York and Philadelphia” \cite[105]{stephenson_patriot_2007}.

Each side had a standard ration supplement for each Soldier.  The American
Army’s ration supply per individual was, “1 lb beef or 3/4lb pork or 1 lb salt
fish per day, 1 lb bread or flour per day, 1 pint of milk per day, 1 quart
spruce beer or cider per day, 3/4 pint of molasses per day, 3 pints of peas or
beans per week, 1/2 pint of rice or 1 pint ‘Indian meal’ per week”
\cite[pp.107-08]{stephenson_patriot_2007}.  The British supply for each Soldier would include, “1
lb beef or 1/2 pork, 1 lb bread or flour, 1/3 pint peas, 1 ounce butter or
cheese, 1 ounce of oatmeal, 1 1/2 gills of rum”
\cite[p.108]{stephenson_patriot_2007}.  Both the
British and American ration cycles delivered plenty of calories, but not enough
vitamins A and C, which could eventually lead to disease.  The loss of New York
created a greater logistical strain on the Colonial Army.

After Greene divided the Southern Continental Army, Morgan was left to find his
own resources.  The majority of Morgan’s supplies before the Battle of Cowpens
came from Grindal Shoals, which was a well-known Pacolet River crossing.  The
camp was a plantation that belonged to a Tory and Loyalist by the name of
Alexander Chesney.  Grindal Shoals was the main Flying Army base until January
14th, 1781.  “During their stay Morgan’s men plundered Chesney’s property of
everything useable including grain, trees, clothing, and blankets.  In his claim
to the British government, Chesney swore the Americans took at least 500 bushels
of Indian corn, in store, a quantity of oats and other crops.  By camping on
loyalist property, Morgan punished Chesney, intimidated other Tories, and
lessened his army’s impact of local patriots.” \cite[p.48]{babits_devil_2001}.  Not all of
Morgan’s troops stayed at Grindal Shoals.  Many detachment forces camped at
Burr’s Mill on Thicketty Creek.  Numerous Cavalrymen received equipment repair
and shoes for horses at Iron Works on Lawson’s Fork.

Tarleton and his forces started at Brierley’s Ferry on Broad River.  He then
moved and halted at Brooke’s Bush River Plantation.  There he requested more
baggage and any additional troops that could be mustered up.  “The area around
Brooke’s had forage, and he anticipated gathering four days’ flour for a move”
\cite[p. 49]{babits_devil_2001}.  This is where Tarleton received more troops and baggage form
General Cornwallis.  He received members of the 7th Regiment and additional
dragoons.  “Tarleton remained at Brooke’s, gathering food and forage, until the
baggage and reinforcements arrived. Everything was in place by the 11th of
January; Tarleton had four days’ food supply, his reinforcements and baggage.”
\cite[p.49]{babits_devil_2001}.  Unfortunately for Tarleton, the supplies he requested were not
enough.  “The four days already spent pursuing Morgan consumed supplies
accumulated earlier at Brooke’s Plantation” \cite[p. 53]{babits_devil_2001}.  Morgan was able to
maneuver his army closer to his supply line.  Because of this, Morgan was able
to feed his men before the battle.  “The Americans had food because cattle were
driven to the Cowpens earlier that day.  Perhaps a battle was not thought so
imminent because Reuben Long, one of the drovers, was discharged early on the
16th of January.  Cattle were butchered that night by James Turner and others”
\cite[p.55]{babits_devil_2001}.  Tarleton’s men would have to fight hungry.

Before the battle each of Morgan’s men were ordered to carry twenty-four rounds.
“By stipulating the number of bullets a man carried, Morgan knew how long a unit
could keep firing and when it should be ordered to the rear before running out
of ammunition.  Envisioning a sequence of linear fire-fights as he drew the
British forward and shot them up, Morgan would evaluate British and American
fighting capabilities as the battle progressed” \cite[p.56]{babits_devil_2001}.   


\subsubsection{Health and Service Support}

Both the British and Continental Armies had extremely similar health
service support.  Each battalion had its own medical surgeon.  The regional
general military hospital existed at the highest echelon followed by a surgeon
at the regiment level.  A government order directed toward Continental forces to
move wounded and sick troops from the regimental medical site to the general
military hospital proved to be a disaster.  Transportation for the patients was
far beyond inadequate and the amount of supplies provided to the regimental
medical site was drastically reduced.  If the regimental medical sites could
have been delegated the authority to maintain these casualties, the probability
of survivability in long term would have significantly increased.   Of the 1200
physicians who served in the Continental Army throughout the Revolutionary War,
only 100 held an M.D. degree.  Most physicians in the military were apprentice
trained.  This was nearly identical in the British Army.  

The medical practices during this time period were very primitive.  For sickness
the medical chest would contain “Peruvian or Jesuit’s bark, potassium nitrate,
and camphor for the treatment of fevers; pain-controlling opium in the form of
gum and tinctures purges like jalap, senna, castor oil, and Epsom salts; emetics
like ipecac and potassium tartrate; red mercuric oxide and mercurial ointment
for the treatment of wounds and venereal disease; and sulfur in hogs lard for
the itch” \cite[p. 170]{stephenson_patriot_2007}.  

	   The basic medical chest for combat would have consisted of “probes
and ball extractors, a range of lancets, splints, saws, and tourniquets for use
during amputations; trepanning instruments; suturing needles; dry and wet
sutures; and bandages” \cite[p.170]{stephenson_patriot_2007}.  Balls would only be extracted if
they were not deeper than an index finger or a ball extractor; if the ball was
too deep in the flesh it would be left in place.  Bones would often be shattered
or ligaments destroyed by the projectiles.  If this were the case, limbs would
be amputated with the use of a compression tourniquet.  “Surgeons who carried
out amputations were taught to show no emotion in the face of screams of the
patient and had to work fast” \cite[p.171]{stephenson_patriot_2007}.   Because of the lack of
antibiotics many patients died from infection.  

Disease played a large part in the contribution of deaths in the Revolutionary
War.  “Of the approximately 25,000 American military deaths during the war,
10,000 died of disease.  Contributing factors were poor nutrition, inadequate
shelter, unsuitable clothing, disregard of hygiene in camp and hospitals,
medical ignorance of the causes and treatment of disease, and concentrations of
young men who had not previously been exposed to diseases”
\cite[p.173]{stephenson_patriot_2007}.
Typhus, scabies, and smallpox are some examples of diseases that were major
killers during this time frame.

\subsubsection{Command, Control, and Communications}

Each Commander had his own personal staff as does any commander today.  Morgan’s
staff officers that served in his headquarters consisted of a brigade major, a
commissary, a quartermaster, and a forage master.  “During the battle, a
Maryland surgeon attached to Morgan’s staff, Doctor Richard Pindell, helped
rally the South Carolina militia before attending to the wounded”
\cite[25]{babits_devil_2001}.  Each of these staff members had a distinct job description, each much
more different than a commander’s staff of today.  These staff members, as
discussed in a previous paragraph, struggled with their duties due to lack of
supplies and equipment, and inadequate means of communication and
transportation.  Majority of staff members did not go through any type of
dedicated training and mostly relied on personal experience and experiences of
senior staff officers.  

The majority of communication to troops was done by drum beat.  "The drum
controlled a soldier’s day.  The drummers in each regiment played different
beats to tell the soldiers where they should be and what they should be doing.”
(The American Revolution: First Phase, 67).  The drum signaled to the army where
to march, which way to face and fire, and when to advance or fall back.  Drums
were used because they were much louder than a human voice and could be heard
above the noise of battle.  Drums were the most efficient way to communicate to
large groups of Soldiers at one time and were relatively efficient.  Lower
ranking officers and non-commissioned officers would be on the front line with
Soldiers directing them where to go often by voice and hand and arm signals.  If
messages needed to be delivered in distance, they would normally be carried by
horseback or by foot.     

 Commanders also had various aides.  These aides were often handpicked and had
good working relationships with their commander.  Aides carried orders and
assisted in administration.  “There were more than simple message carriers; they
spoke with the authority of Morgan himself” \cite[24-25]{babits_devil_2001}.  These aides
have many similarities to the operations officers of commanders today.  Like an
S3, any order that would come from these aides to a lower echelon commander
would be like an order coming from the higher echelon commander.  These aides
often delivered important messages to other commanders before the battle would
begin.

Operation orders and plans were not thought out by a commander’s staff as they
often are today.  Commanders relied on personal experience and may have received
input from lower echelon commanders on the plan and operation to be conducted.
These plans would be developed by reconnaissance of terrain, movement of the
enemy, and resources provided to their troops.  Reconnaissance of terrain would
often be conducted by a small scout team consisting of a middle to higher
echelon officer or the commander himself.  Intelligence of enemy movement would
be updated to the commander through his scouts.  These updates were often
brought to the commander on horseback, which was the fastest way of
communication at the time.

The extent of transmission security was a cipher.  American and British forces
employed codes and ciphers to disguise their communications, and took
precautionary measures to ensure that crucial messages were not intercepted by
the enemy. Both armies employed replacement codes, where pre-set letters or
words replaced other letters or words in communications.  At first these ciphers
were fairly easy to decode.  Later in the war American scientists invented and
invisible ink that could be used to carry messages.    

\subsubsection{Intelligence}

Before the Battle of Cowpens both commanders of their units had ways of
gathering intelligence on their respective opponents and terrain in the area.
The majority of their intelligence came from locals who lived throughout the
area or scouts the commanders would send for reconnaissance.  “Earlier that
afternoon, Morgan went ahead and met with local residents before the Flying Army
reached Cowpens.  Before deciding to fight, he conducted a reconnaissance the
afternoon of 16 January.  Captain Dennis Tramell recalled that, “the
Cowpens…being in two and a half miles of his residence…and he being well
acquainted with the local situation on the ground…with General Morgan and his
life-guard Aide d camp went out and selected the ground upon which the Battle
was fought”” \cite[p. 53]{babits_devil_2001}.  

Reconnaissance of enemy troops was also conducted by scouts who would either get
approached by or talk to local residents.  “Tarleton continued to collect
information and plan for the next day.  He used his own American well, including
Alexander Chesney.  “To get information on Morgan’s men, he sent me out… I rode
to my father’s who said Morgan was gone to the Old-fields about an hour before….
I immediately returned to COL Tarleton and found he had marched towards the Old
Fields.  I overtook them before 10 o’clock… on Thickety Creek””
\cite[p.56]{babits_devil_2001}.
Prisoners were also interrogated for information when captured.  “Tarleton also
interrogated at least one prisoner who claimed to be an American militia
Colonel” \cite[p.56]{babits_devil_2001}.

Both Commanders used information they gathered to make decisions before the
battle.  The information gathered determined where the battle took place and
when both Commanders decided to move their troops into position.  Any
information gathered was taken directly to the commander if it was critical
enough.  This information was usually brought to him from a scout on horseback.
“A man on horseback, even moving discreetly in the night, could cover the twelve
miles from Burr’s Mill in less than an hour” \cite[p.57]{babits_devil_2001}.  From there, each
commander would make a decision and disseminate it either through his respective
staff or lower chain of command.  There was no intelligence officer to war-game
at the time; all war-gaming was done by the commander. 

\subsubsection{Information Operations}

Whether it was a campaign to gain support or spies with information, all
information operations played a part in the war.  The impact of information
operations was more evident in the South as it was crucial in influencing a turn
in positive momentum for the American forces. Lieutenant Colonel Tarleton was to
blame for much of the propaganda used to feed resentment towards the British
throughout the American southern colonies. During some of the southern
skirmishes it was said that Tarleton was ruthless even to a surrendering army.
“His destruction of Beauford’s command at Waxhaws, South Carolina, and the
infamous brutality of his officers and men toward wounded and prisoners there
and elsewhere, created an impression of savagery that served both to enhance his
operations and rally the opposition” \cite[p.44]{babits_devil_2001}.  It was said in one
instance that he did not take any prisoners of war and instead killed them after
they laid down their arms. This enraged the Colonials as well as some of the
British.  The British knew that this could and would be used against them to
recruit more soldiers for the Continentals and win more civilian support.  

The message was clear, put in paper, and of course spread throughout the
colonies.  It may have been made into a bigger ordeal than it actually was in an
attempt to gain support for independence, aid in recruitment for the militias
and Continental Army, and increase support for the Continental Army from
Congress and the local populations.  In the end the exploitation of Tarleton’s
brutality at Waxhaws was successful. The morale of the Continentals began to
brighten because of a growing confidence that recruits, supplies and funding
would all increase in numbers. An increase in favor for the Continental Army and
disgust towards the atrocities of Tarleton and his men would assist in the
downfall of the British efforts in the South.

Tarleton received information on the rebels from British loyalists who were
eager to share information on enemy leadership locations and troop movements. “I
have sat down to acquaint you with what I have heard a few moments ago Morgan
and Washington had joined the party that lay at Grimes Mill yesterday and they
all moved to Colonel Henderson Plantation about a mile this side of the mill and
I well informed that they intended to March as fast as they can to Ninety Six I
don’t believe they have as much men as it is reported to my wife’s sister”
\cite[p.182]{moncure_cowpens_1996}.  This letter was sent to Tarleton’s camp from a British
sympathizer.  In actuality the information was inaccurate.  The British were
plagued with misinformation, but often acted on intelligence that was rarely
validated.  Erroneous information continually stressed the British commanders
and often forced them to reallocate military assets based on poor intelligence
gathered or provided from the local populace.   

With the news of a defeated British army at the Battle of Cowpens the tide of
the war changed forever in favor of the Americans. There was a renewed,
invigorated hope in the Continentals and an increased support to finish off the
remaining British forces to end the war.
\subsubsection{Tactical Doctrine and Training}

“While Morgan was undoubtedly concerned about his troops’ ability to meet the
British on equal terms, the British were not unduly concerned about American
tactics. They were concerned about the American rifles they knew had greater
accuracy and range that made their fire dangerous at a distance”
\cite[p.18]{babits_devil_2001}.
British tactics generally consisted of soldiers in a rank and column when
fighting against regular infantrymen. These tactics along with the combined arms
of artillery and cavalry made the British the strongest fighting force in the
world. In the south Tarleton had his army fighting in a loose formation which
allowed for greater frontal coverage, less damage and made it easier for command
and control in wooded areas.  To get the most fire power down range, soldiers
would send a volley of musket fire into the enemy formations.  This was used to
kill and create shock and awe on the battlefield.  It could also be used to
deter a bayonet charge. Though these tactics were effective, it did lack depth.
After the first initial volley, soldiers on the British side would fire in
planned sequences to ensure a continuous fire, usually alternating between
platoons or divisions.  

The Americans, like their counterparts, started fighting in linear formations.
The Americans lacked the discipline of the British and were too loosely
connected to each other causing formations to disintegrate and retreat after the
first volley of fires.  Both armies were broken up into sections comprised of
battalion and regiment sized elements.  As the war progressed, the Continentals
knew that the British had an advantage in strength in numbers. The Americans
would eventually partake in guerilla type tactics in an effort to weaken the
British forces.  This is evident throughout the Southern Campaign under the
command of General Nathanael Greene and his directive to Brigadier General
Daniel Morgan, 

\begin{quote}
“Greene’s instructions to Morgan are extremely interesting because they remain
excellent advice to a guerrilla column operating through enemy territory in the
twentieth century as well.  Greene wrote that the object of Morgan’s special
force was to protect the part of the country in which it operated, annoy the
enemy, ‘spirit up’ the people, collect provisions, forage out of the way of the
enemy, prevent plundering, and give receipts for whatever was taken, at least to
all friends of the American cause” \cite[p.120]{lumpkin_savannah_1981}.
\end{quote}

The transformation towards guerilla tactics enabled the Continental Army to
survive and rebuild after many early losses to a well-trained and organized
British Army.  The British soldiers were not used to seeing guerilla tactics and
were unable to successfully adapt to a different type of warfare.  After General
Greene assumed command of the Southern Continental Army, his forces would
eventually meet the British again on a more formal battleground.  His commander
(Morgan) at the Battle of Cowpens would introduce new tactics to counter the too
familiar methodologies of the British Army.  

British soldiers possessed far superior training in comparison to a mostly
militia sourced Continental Army. “Justifiably, the 71st Highlanders were
regarded as first-rate troops” \cite[p.45]{babits_devil_2001}. The Continental Army had a weak
logistical infrastructure, which prevented them from outfitting the attached
militiamen with the items necessary to properly train and organize a respectable
fighting force.  The weaknesses of the militias were evident to the American
commanders, but their involvement was essential to the continued fight for
independence, “…he [Morgan] had no illusions about the behavior of militia in
formal battle.  But the use of militia in battle was vital to the cause because
there were rarely enough Continentals to face the British alone”
\cite[316]{buchanan_road_1997}.  Morgan also understood that the militia was an asset if used properly,
“He would not try to get militia to do what they were not meant to do.  For he
knew them.  He came from them, those country people and backwoodsmen, knew their
faults and virtues, their capabilities and failings, knew as did William
Moultrie that ‘the militia are brave men, and will fight if you let them come to
action in their own way’” \cite[p.316]{buchanan_road_1997}.  

\subsubsection{Condition and morale}

Morale amongst the British Army was in continual decline as the Southern
Campaign continued to drag on and failed to produce any meaningful victories.
Cornwallis began to realize that success in the South would be more difficult
than originally presumed, “Lord Cornwallis revealed extreme pessimism, even
disgust, with regard to the courage and abilities of the Tories of the
Carolinas, a group that was the linchpin of British strategy in the South”
\cite[p.306]{buchanan_road_1997}.  Cornwallis would eventually write to his commander, Sir
Henry Clinton on January 6, 1981, “…in the gloomiest of terms the true situation
in South Carolina.  ‘But the constant incursions of Refugees, North Carolinians,
and Black-Mountain men, and the perpetual risings in the different parts of this
province; the invariable success of all these parties against our militia, keep
the whole country in a continual alarm, and renders the assistance of regular
troops everywhere necessary’” \cite[pp.306-7]{buchanan_road_1997}. 

While in pursuit of General Morgan, the morale of the British would only worsen
due to weather and poor foraging opportunities. “The cumulative effect of short
rations, lack of sleep, harsh marching cold, wet weather, and the fighting to
this point left them unable to continue” \cite[93]{babits_devil_2001}.  This made the
British only question their belief of the cause at hand, as well as some
impatient politicians who would challenge the king on the cause of the war. 

Tarleton knew he had the more superior force, but his men lacked the care that a
good leader would provide.  His men were physically and mentally drained.
Tarleton may have believed that a victory over Morgan would refresh the morale
of his men and shift the tide of war in favor of Britain.  Mostly all of these
men were seasoned veterans of the war. A fleeing or retreating Continental army
could reignite a sense of purpose and accomplishment among Tarleton’s men.   

The Continentals Army was not much better off.  In almost every battle they were
outmatched and outnumbered. Their discipline was low as soldiers would just run
away in the night. They were cold, wet, and supplies were running thin.  Morale
was struggling.  Many of the recent skirmishes ended in defeat.  Various units
in Morgan’s command had seen some fighting, but a large part of the militiamen
had no previous fighting experience.  General Morgan was able to inspire esprit
de corps within his formation prior to his meeting with Tarleton at Cowpens.
Morgan rested, fed, and socialized with his men the night before the Battle of
Cowpens creating an physical and mental advantage over Tarleton who would push
his men to meet with Morgan with little food or sleep.  “Morgan’s preparations
throughout the night were not in vain.  His men were fed and resting in line of
battle on the ground of his own choosing.  They knew what was expected.  Morgan
was ‘in a popular and forcible style of elocution haranguing them’”
\cite[p.60]{babits_devil_2001}.  Reinforcements also began arriving throughout the night
and further elevated a sense of confidence in the Continental soldiers that
would face a British army the following morning.   The popularity of the
rebellion in the south also produced an increase in morale with the soldiers,
“Cornwallis himself admitted that the spirit of rebellion was alive and well in
South Carolina” \cite[p.307]{buchanan_road_1997}. 

During the battle of Cowpens the British moral was on the rise with the thought
of a quick victory in sight. Though that victory would soon enough fade as the
Colonials fought back with their main force and the tide of the battle shifted.
The British being overrun frantically ran away, their discipline arrayed, and
the moral of the Continental soldiers were up. Lieutenant Colonel Banastre
Tarleton felt the shift in momentum of the battle and barely escaped with his
life. 

\subsubsection{Civil Affairs}

In the southern campaigns civilians tried to avoid the fighting. When Charleston
was lost, everyone except the Loyalists became prisoners.  Civilians perceived
to be supporters of the rebellion were made to give up their weapons and give
housing to the British soldiers and Loyalists.  During this time the King gave
the civilians an ultimatum, support the British or be judged as a rebel. Those
loyal to the crown used this as an excuse to pillage their neighbors who were
not Loyalists and took their lands. Tory militias would terrorize those believed
to not support the King.  “Some 250 Georgia Tories under Colonel Francis Waters
were raiding Patriot settlements only twenty miles south of Grindal’s Shoals, in
the area of Fair Forest Creek below modern Spartansburg”
\cite[p.302]{buchanan_road_1997}.
Such actions of course, created campaigns of retribution from the American
forces.  “If he [Morgan] was to ‘spirit up the people’ he had to defend them.
Soldiers would also try to mitigate civilian hostilities by writing IOU’s on
supplies that they used or borrowed. Morgan added 200 mounted militiamen under
Major James McCall to William Washington’s dragoons and ordered the hard-riding
cavalryman to advance against the raiders” \cite[p.302]{buchanan_road_1997}.  The revenge that
William Washington exercised on the Tory militiamen when he caught up to them
was striking, “In one of the war’s more brutal actions the Rebels rode into the
fleeing ranks of their fellow Americans and hacked at them without mercy…The
Tories that were not killed were horribly slashed and mangled by the big cavalry
sabers” \cite[p.302]{buchanan_road_1997}.  

\subsubsection{Law of War / Enemy Prisoners of War}

The established law of war differed throughout the British and American ranks
during the Revolutionary War. There was acknowledgement of laws, but it was not
fully followed by either side. The British were more prone to abuse their
prisoners. In the south, prisoners of war (POW) in general were not treated very
well and the conditions in which they were held were horrific. “Scholars believe
that at least 8,500 of the 18,154 Continental soldiers and sailors who were
captured in this war died while in captivity” \cite[p.428]{Ferlin}.  This was in
part because the British leadership did not ensure that prisoners were well
treated and cared for.  In actuality, it does not appear that they ever
exercised a true concern. The Americans were held in many places and different
facilities. Some of the most horrific detaining facilities were prison ships.
Prison ships were dark, cramp, and stuffy.  It was said that being in one of
these ships was like being sentenced to death.  Many diseases were spread
through air and through feces. 

Some soldiers were never provided an opportunity to surrender peacefully. Under
Tarleton’s command any enemy combatant that surrendered ended up dead.  He held
nothing back. The prisoners taken by the Continentals were treated fairly.
General Morgan in a letter to General Nathaniel Greene wrote that he had taken
502 NCO’s and privates and 29 officers as POW’s.  SGM Williams account of POW’s
was slightly more on the numbers of enlisted men taken prisoner. He also
reaffirmed to General Greene that “not a man was killed, wounded, or insulted
after he surrendered” \cite[p.122]{moncure_cowpens_1996}.  General Morgan took his prisoners to
Salisbury and the officers were paroled. General Morgan did not want to revert
to the rumored barbaric ways of Tarleton. 

\subsubsection{Leadership}

Two of the strongest colorful personalities in the War of Independence meet in
the south at a pasture called Cowpens. General Morgan and Lieutenant Colonel
Tarleton though on opposing sides had much in common. Both were respected by
their soldiers and feared by their enemies. Leadership was a defining aspect in
this battle.  Morgan was a well experienced, seasoned veteran, who knew how to
employ his assets and knew the strengths and weaknesses of his men.  Tarleton
was a determined individual who was impatient and ruthless. Though these two
arguably played the most important roles at the Battle of Cowpens, the roles of
their superior and subordinate officers were crucial. 

Though not engaged in The Battle of Cowpens, it was General Nathaniel Greene who
sought out and gave General Morgan free reign to fight or flee the British Army
as he saw fit.  Greene knew that splitting his forces in the south would have a
greater impact on fighting the British. Greene’s assessment of his and the
British forces provided him insight that dividing his army would afford him the
needed time to fight the British on his terms. General Greene came from a Quaker
background and at age 32, was the army’s youngest general. He was appointed
Brigadier General of the Army of Observation out of Rhode Island.

During a stalemate in the northeast, he took the time to learn from the
experiences of veteran soldiers and took to reading books. Greene helped General
George Washington plan and attack on Trenton from which he gained Washington’s
respect. Greene was also involved in the night offensive on Germantown where he
gained more notoriety. Greene was a very selfless leader, which was evident in
his desire to not receive a general’s salary during the tough economic times of
the Revolutionary War. Greene was a very capable and competent leader who would
help counsel Washington when he sought advice.  Washington placed him in charge
of the Southern Continental Army when the first two commanders failed. 

Brigadier General Daniel Morgan was second in command of the southern troops.
“Living and working in the roughest societies, Morgan was noted as a tough man
among very tough men” \cite[p.116]{lumpkin_savannah_1981}.  While growing up he was a hard drinker
as well as a womanizer. What defined Morgan was his love to fight and in
unconventional ways. This characteristic made Morgan the perfect person to
engage the ruthless Tarleton. 

Morgan’s experience dated to his involvement in the French and Indian War where
he fought alongside with the British.  Morgan applied what he had learned during
that war and applied it to how he fought the British during the Revolutionary
War.  Morgan did not have any formal military training, but he did devise his
own unique style of fighting and incorporated guerilla tactics learned in the
French and Indian War. Morgan’s finest hours came during the Battle of Cowpens.
He understood the British and used it to his advantage.  Morgan knew that if he
had his men appear to retreat during the Battle of Cowpens, the British would
hastily pursue in an attempt to destroy the Continentals before they could get
away.  Morgan was a man who  understood his men and grasped tactical concepts
very quickly.

Colonel Andrew Pickens came from a Scottish-French background that followed his
family from Pennsylvania to Virginia.  At the latter end of his teenage years he
was left fatherless, so he decided to follow in his father’s footsteps and
become a militiaman.  Pickens was a bright young man and also very mindful of
Britain’s rule.  Pickens was assigned before the battle to examine the soldiers
who were prisoners of King Mountain.

Pickens had previous battle experience form fighting in the French and Indian
War and the Battle of Kettle Creek. He adapted his guerilla tactics from this
war and would later use it to fight against the British in the south.  In 1780,
Pickens accepted protection from the British and tried to remain neutral.  After
his family’s plantation got pillaged, he could no longer remain neutral.  He
sided with General Morgan and thus began the downfall of the British in the
south. Pickens was a very knowledgeable counsel to Morgan.  He aided in Morgan’s
decision to remain at Cowpens and engage the British.  

Lieutenant Colonel John Howard was from Baltimore and considered a superb
officer by others who served with him.  Howard established himself as a fearless
leader in battle.  He had previous experience from serving in the northern
campaigns. Nathaniel Greene would later write “Howard as a good officer as the
world affords. He has the great ability and the best disposition to promote the
service…. He deserved a statue of gold.” \cite[26]{babits_devil_2001} Howard obviously was very
well respected by his command. 

Lieutenant Colonel William Washington was very athletic and a very skilled horse
rider. He would be later assigned to the 4th Continental Light Dragoons, who saw
action in Brandywine, Germantown and in the Monmouth campaigns. Washington was
then sent to the south to help in the efforts of the rebellion and placed in
charge of the 3rd Dragoons.  He would frequently encountered run-ins with the
British cavalry led by Tarleton.  Washington’s quick thinking and ability to
adapt to any situation aided him when he made the Tories surrender  using a fake
cannon. Washington was able to do more with less.  He would accept any challenge
even if it appeared to be more than he thought he could handle.

General Charles Cornwallis was a strong-minded and strong-willed individual.  In
the American War for Independence he would be the leader that gave America the
most problems in the south.  Cornwallis was very loyal to King George and was
amongst the king’s favorites.  Cornwallis was very sympathetic to the colonies
and did not always agree with the way King George would conduct business. When
push came to shove with the colonies, Cornwallis served the king admirably in
the war of Independence. Cornwallis was a veteran to warfare.  He fought in the
seven year war against Germany and battles in the southern colonies.

Sent to fight in the south, Cornwallis was quick to achieve victory in most of
the battles he was in.  Of course, having the greatest fighting force in the
world also helped.  He would eventually underestimate the Continental and
militia armies on their abilities to maintain momentum throughout the Southern
Campaign.  When in pursuit of General Greene, Cornwallis made the error in
dividing his forces and following Greene deep into the back country of the south
away from his supply lines. Despite the Continental Army’s continuous use of
guerilla tactics, Cornwallis never changed his strategy or requested a revised
mission from his commander. 

Lieutenant Colonel Banastre Tarleton was much of a loose cannon for the British.
Tarleton’s professional background was in law.  He was only 26 and had combat
experience from battles in Charleston and New York.  After his campaigns in the
south he earned the name “Bloody Tarleton” \cite[p.44]{babits_devil_2001} due to his brutality
towards POW’s and even his own officers.  Tarleton was a leader whose merciless
attitude and persistence had served him well throughout much of the American
Revolution.  This however, would eventually lead to his failure at the Battle of
Cowpens, where his aggressiveness, arrogance, and high and mighty attitude got
outwitted by General Morgan. 

Tarleton was blinded by the idea of a quick victory and did not hold back his
troops from perusing what seemed to be a retreating militia army. What seemed as
a sure victory quickly turned into a devastating loss for the British and would
swing the military momentum in favor of the Americans. Tarleton and Cornwallis
both underestimated the determination and the resolution in their American
military leaders as well as the spirit to fight that still existed in the
American soldier. 

\subsection{Feasible Courses of Action}

The tactical courses of action for both the Continentals and British were in
many ways dictated by customs and knowledge of tactics of the era, as well as
the terrain.  Morgan, performing a tactical retrograde away from Tarleton toward
the broad river, was limited in his options.  He could fight a battle against
the British, continue moving toward the river and hope to stay ahead of them, or
use his forces to harass the enemy without becoming decisively engaged.
Continuing to move toward the Broad River would likely have been problematic at
best: “In January 1781 conditions for both armies in the Carolinas could hardly
have been worse.  A deluge fell on the land.  What were bad roads in dry weather
became much and mire under the incessant heavy rains that not only flooded
rivers and creeks and branches but the land itself for miles beyond.  On one
occasion, Greene informed Morgan, the Pee Dee rose twenty-five feet in thirty
hours.  Faithful dispatch riders on both sides urged their jaded horses through
a nightmare of water and mud” \cite[p.310]{buchanan_road_1997}.  With his escape route becoming
more and more hazardous Morgan chose the first option, “with the broad river
behind and to the east of his force, he chose to face Tarleton.  Morgan chose
his ground wisely.  He anchored his defense on a low hill and arrayed his
infantry to the south” \cite[32]{brinkley_back_1998}   

Once that choice was made, Morgan ran into a completely different problem: how
was he to array his forces to face the British?  He had the option of sticking
to tradition, but he had noticed that ``repeatedly in earlier battles,
inexperienced militia had ruined everything for the Revolutionaries by fleeing
the field” \cite[30]{weigley_partisan_1970}.  With this in mind, he chose instead to “deploy
progressively stronger infantry to shoot up the British as they advanced”
\cite[71]{babits_devil_2001}.  Utilizing different types of assets in a combined arms fight
was nothing new, but Morgan added a twist: he integrated the militia’s tendency
to run away into his plan, and used it to his benefit.  

Tarleton's choices were similarly dictated by current techniques of battle, as
well as by the necessity to react to Morgan's activities.  The broad choices he
had were to hound the Revolutionaries and force a confrontation, or to forego
the chase and lay an ambush in a place of his choosing.  Tarleton chose to ``hang
upon General Morgan's rear to cut off any militia reinforcements that might show
up'' \cite[46]{fleming_cowpens_1988}.  He hoped to catch Morgan as his forces were crossing the
Broad River and claim a complete victory against the American forces.  On the
smaller, tactical scale, the British commander chose to form up as custom and
tradition demanded upon the battlefield:  ``as Tarleton had done in the past, he
sent the legion cavalry to disperse the militia and gain information of Morgan's
dispositions'' \cite[33]{brinkley_back_1998}.

In respect to understanding the complete situation, Morgan had a much greater
feel for the terrain, friendly forces, and enemy forces than Tarleton did.  He
adjusted his tactics to compensate for the propensity of the Militia to cut and
run when pressed, “Therefore, Morgan in an act of shear (sic) genius, arrayed
his militia in the first rank with a line of forward skirmishers.  His
Continentals constituted the third rank.  Morgan’s orders to the militia were
simple and clear: fire two or three well aimed shots then move to the rear and
flanks as a reserve'' \cite[32]{brinkley_back_1998}. Tarleton, though he did have a good
understanding of the terrain, did not see any need to adjust his tactics as they
had always served him well, and proceeded to face his opponent with standard
battle lines.  This attitude was enforced by Morgan: ``Morgan's trap depended on
breaking down the British, and, if Tarleton could not see the American lines,
the later appearance of new, stronger lines would come as something of a
surprise'' \cite[82]{babits_devil_2001}.  The limited visibility of the morning and inability
for a full reconnaissance kept the fact that Morgan interspersed his militia
with skirmishers from the British, and did not allow Tarleton  a full
understanding of the battle that was about to commence.

\subsection{Tactical Missions of the Antagonists}


Morgan had been given a great deal of leeway in his mission from General Greene,
he was given “orders to conduct himself either offensively or defensively, as
your own prudence and discretion may direct – acting with caution and avoiding
surprises by every possible precaution” \cite[27]{weigley_partisan_1970}. Tarleton’s orders were
to “pursue Morgan and either destroy him or force him to retreat over the Broad
River again” \cite[30]{fleming_cowpens_1988}. Based upon his chosen course of action, Morgan’s
likely tactical missions were to occupy the key terrain in the area, secure his
troops, then canalize and destroy the British forces.  Tarleton’s mission set
would have included attacking by fire to defeat enemy forces and neutralize
their capability for a counterattack or to conduct follow on operations.  Both
the Continental and British Commanders’ tactical missions were consistent with
the orders they received from their superiors.  
