\section{The Strategic Setting}
\doublespace
\subsection{Causes of the Conflict}

\subsubsection{--text--}

There are many factors of politics, trade, economics and policy that all
affected the colonies break away from Britain One analysis of the American
Revolution sees it as a continuation of the Seven Years War. Another focuses on
its nature as a revolt against tightening of mercantilism around the colonies,
who were slowly gravitating towards idas of trade codified by Adam Smith's
``wealth of Nations,'' published in \red{1776}. The popular analysis lies in a
revolt against laws seen as unjust, levied without representatoin. This section
will address in turn the previous war, mercantilism, the influence of other
foreign powers, and finally the series of political events that drew a line in
the sand for the colonies.

\red{The American Revolution, from the international standpoint, was one in a series
of conflicts between Europe's major powers that also took part in America.
Britain and France had already been at war off an on for \red{????} years before
the Revolution, both in Europe and in France. }

\red{\emph{ [[[Talk about the Intercolonial Wars]]]}}

The political and economic origins of the American revolution are tied to King
George III's resentment of the Whig parliment that had ruled England during the
reign of his father and grandfather. George III came to the throne as a native
Brit, rather than a German import, and rather than allowing parliment and the
Prime Minister to govern, he was determined to replace the Whigs with Loyalists
and restore power to the throne. The political maneuvering of his followers,
the leadership of William Pitt, the popular Whig secretary of state and
victorious strategist of the Seven Years War, and the `salutary
neglect'  policies of his predecessor Sir Robert Walpole, Prime Minister under
King George the I.\cite[Ch 1-4]{cook_long_1995}

Walpole ``took the enlightened view that if the colonies were left to run their
own local affairs with minimal interference from London, they owuld produce more
wealth and commerce, prosper, and give less trouble.''\cite[p. 3]{cook_long_1995}

George Grenville


\subsubsection{--notes--}

\singlespace
\begin{singlespace}
\ctable[
  cap={Events leading to the Revolution},
  caption={Timeline of events leading to the American Revolution.},
  pos=h,
  ]{>{\footnotesize}r>{\footnotesize}p{4in}}
  {
    \tnote{ Sources include: \cite[323-326]{berard_1878}, \cite{cook_long_1995}.
    }
  }{\FL
Year 		& Event 									\ML
%~		& \red{Currency Act:}								\NN
%~		& \red{Quartering Act:}								\NN
%~		& \red{Intolerable Acts:}							\NN
%~		& \red{End of Sir Robert Walpole's `Salutary Neglect' policy (Cook, 3)}		\NN
%~		& \red{Boston Tea Party}							\NN
%1765		& \red{Stamp Act}								\NN
%Apr. 1764	& \red{Sugar Act:} passed prompting cries of ``Taxation without Representation''\NN
%Jan. 1775	& Benjamin Franklin departs London a `determined revolutionary' (Cook, xii) 	\NN
%1776		& \red{Adam Smith publishes the \emph{Wealth of Nations}}			\NN
%Sep. 1783	& The American Revolution ends with the treaty of !!!				\NN
%~ 		& ~ 										\LL
%
%
%1749 		& Commandant C\'eloron lays claim to the Ohio Valley for France 		\NN
%\red{Nov. 1749}	& ?? Navigation Acts								\NN
1754		& French begin the French and Indian War in the states, which
		  spreads to Europe to become the Seven years War.\NN
%Jan. 1757	& Benjamin Franklin travels to London as a Colonial representative		\NN
Oct. 1760	& King George III ascends to the throne of England				\NN
1761		& Writs of Assistance\NN
%Nov. 1761	& James Otis challenges the Writs of Assistance, allowing searches to prevent
%		  smuggling, as violations of the constitutional rights of colonists.		\NN
1763		& Treaty of Paris ends the Seven Years' War\NN
Oct. 1763	& Proclamation of 1763: King George III forbids colonists settling west of
		  the Appalachian Mountains.							\NN
Mar. 1765	& Stamp Act\NN
Oct. 1765	& Colonial Congress\NN
Mar. 1766	& Stamp Act Repealed\NN
Oct. 1768	& Troops sent to Boston\NN
1770		& Boston Massacre\NN
16 Dec. 1773	& Boston Tea Party\NN
Sep. 1774	& Continental Congress\NN
19 Apr. 1775	& Battles of Lexington and Concord\NN
4 July 1776	& Declaration of Independence\NN
1777		& {Washington at Valley Forge}\NN
1778		& {French Treaty.}\NN
29 Dec. 1778	& Savannah captured.\NN
12 May. 1780	& Loss of Charleston\NN
17 Jan. 1781	& \bfseries{Battle of Cowpens}\NN
15 Mar. 1781	& Battle of Guilford Courthouse\NN
19 Oct. 1781	& Cornwallis's Surrender\NN
19 Apr. 1783	& Cessation of Hostilities\NN
3 Sep. 1783	& Treaty of Paris\LL
}
\end{singlespace}

%---------------------
% % http://militaryhistory.about.com/od/americanrevolution/a/amrevcauses.htm
% Rise of Liberalism and Republicanism
% 
% As tensions regarding colonial lands and taxation increased during the 1760s
% and 1770s, many American leaders were influenced by the liberal and republican
% ideals espoused by Enlightenment writers such as John Locke. Key among Locke's
% theories was that of the ``social contract'' which stated that legitimate state
% authority must be derived from the consent of the governed. Also, that should
% the government abuse the rights of the governed, it was the natural
% responsibility of the people to rise up and overthrow their leaders. The ideas
% of Locke and other similar writers contributed to the American embrace of
% ``republican'' ideology in the years before the Revolution. Standing in
% opposition to tyrants, republicanism called for the protection liberty through
% the rule of law and civic virtue.
%--------------------------


%\doublespace

The War of Grand Alliance and King Williams War 

In the War of Austrian Succession (1740-1748) Britain and Austria allied against
France, Spain and others, while in the Americas, Britan's colonies fought the
French and Indians in King George's War (1744-1748) \footnote{Rickard, J. (22
October 2000), King George's War (1744-18 October 1748),
\url{http://www.historyofwar.org/articles/wars_kinggeorge.html}}


\begin{quotation}\red{
``George III was twenty-two years old when he began a history-laden reign that
would last for sixty years. He was headstrong and obstinate, full of youthful
prejudices, lacking in humility, wisdom, or vision, with little interest in the
opinions of others. But he was also hardworking, prudish, religious, and moral.
He came to the throne determined not merely to reign but to rule. To do so, he
would have to bring his own Loyalists to power and open a new political era
after a half century of dominance by the Whigs. He judged that England was
war-weary and overburdened with debt. He wanted peace, and he took no great
pride in Pitt's faraway victories in the American colonies or other parts of the
globe. American trade was booming, there were no signs of trouble, and all
seemed well.'' \cite[p. 5]{cook_long_1995}
}\end{quotation}

\begin{quotation}\red{
``A great Empire and little minds go ill together,'' Edmund Burke in the House
of Commons. March 1775
}\end{quotation}

\begin{quotation}\red{
``Historians generally contend that once the British Army had cleared the French
out of North America and the colonies were no longer threatened by this outside
power, American independence became inevitable.\ldots'' altough Cook explains
that this was not inevitable when George III first began his reign, before the
end of `Salutory Neglect` and the increased direct rule.\cite[p. 5]{cook_long_1995} 
}\end{quotation}

\begin{quotation}\red{
``At the end of the Seven Year's War, her merchant fleet totaled some 500,000
tons of sailing ships.''\cite[p. 15]{cook_long_1995}
}\end{quotation}




\footnote{
  February 6, 1765:

  ``They, planted by your care? No! Your oppression planted 'Em in America,
  They fled your tyranny to an uncultivated an dunhospitable country where they
  exposed themselves to almost all the hardships to which human nature is
  liable, and among others to the cruelty of a Savage foe. And yet, actuated by
  the Principles of English Liberty, they met all these hardships with pleasure,
  compared with those they suffered in their own Country, from the hands of
  those who should have been their friends.''

  ``They, nourished by your Indulgence? They grew up by your neglect of 'Em; as
  soon as you began to care about 'Em---sent to spy ou their Liberty, to
  misrepresent their Actions, to prey on 'Em.''

  ``They, protected by your Arms? They have nobly taken up Arms in your defense,
  have exerted a valour against their consent and Laborious industry for the
  defense of a Country whose frontiers, while drenched in blood, its interior
  Parts have yielded all its little Savings to your Emolument. And believe me,
  and remember I this day told you so, that some spirit of Freedom which
  actuated the at people at first will acompany them still\ldots
  
  ''I claim to know more of America that most of you, having seen and been
  conversant in that Country. The People I belive are as truly Loyal as any
  Subjects the King has, but a people Jealous of their Liberties who will
  vindicate them, if ever they should be violated\ldots''
  
  Colonel Isaac Barr\'e during debates in the House of Commons on the Stamp act
  on Feb. 6, 1765, in reply to Charles Townshend's rhetoric describing the
  Americans as ``children planted by our care, nourished by our indulgence\ldots
  protected by our arms'' who were begrudging England their rightful returns for
  protection and defense of the colonies.\cite[p. 67]{cook_long_1995}
} 

%%% --begin-notes
\subsection{--notes--}\footcite[p. 2]{moncure_cowpens_1996}

Enlightenment political theory:

French and Indian War (1754-63)

Village green in Lexington, MA 19 APR 1775

trade unrestricted prior to 1760s

King sends:\\
Major General Sir William Howe\\
Major General Sir Henry Clinton\\
Major General Sir John Burgoyne

\paragraph{--notes--}\footcite[]{}
``Shy. John, \emph{A People Numerous and Armed.} pp. 198-199 and 230-232. The British contemplated cutting loose New England and retaining the richer lands of the South in 1778. but never pursued a negot~ated settlement with the Continental Congress.''

%%% --end-notes

\begin{quotation}
  What do we mean by the Revolution? The war? That was no part of the Revolution;
  it was only an effect and consequence of it. The Revolution was in the minds of
  the people, and this was effected, from 1760 to 1775, in the course of
  fifteen years before a drop of blood was shed at Lexington. The records of
  thriteen legislatures, the pamphlets, newspapers in all the colonies, ought
  to be consulted during that period to ascertain the steps by which the public
  opinion was enlightened and informed concerning the authority of Parliament
  over the colonies.\footnote{John Adams to Jefferson, 1815}
\end{quotation}

\subsection{Mercantilism}

\subsection{--notes--: Cook}

King George III ascended to the throne in 1760. He succeeded two German
speaking kings, and was the first native-born soverign over England in
\red{???} years. Along with these other differences, he was at odds with the
Whig party which has previously driven politics and policy in England for
\red{??? years}. His transition was three years prior to the end of the Seven
years war, which ran from \red{???} to 1763. And, although crowned following a
series of victores, George III's first proclamation as ruler reflected his
desire to end the war, rebuild a war-burdened economy and country that had
already reached it's highest level of debit so far.


