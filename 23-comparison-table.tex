%\documentclass[letterpaper,12pt]{article}
%\usepackage[margin=1in]{geometry}
%\usepackage{xtab}
%\usepackage{ctable}
%\usepackage{enumitem}

% Overview text goes here.
%%\begin{document}
%  \ctable[
%  	cap=foo,
%	caption=Elements of National Power,
%	]{l>{\tiny}p{2.7in}>{\tiny}p{2.7in}}{}{\FL
%%\begin{table}[h]
\begin{singlespace}
\footnotesize
  \tablecaption{Comparison of the antagonists' elements of national power.}
  %\tablecaption{Comparison of the Antagonists}
  \tablefirsthead{\FL\centering{USA} & \centering{Great Britain}\ML}
  \tabletail{\ML\multicolumn{2}{c}{\centering{continued\ldots}}\ML}
  \tablelasttail{}
  \tablehead{     \multicolumn{2}{c}{Comparison of the Antagonists, (cont)}\NN\centering{USA} & \centering{Great Britain}\NN}
  \label{table:comparison}
\begin{mpxtabular}{p{3in}p{3in}}%\FL
  %\centering{USA} & \centering{Great Britain} \ML
  \multicolumn{2}{c}{Diplomatic}\ML
  \begin{itemize}[nolistsep,leftmargin=*]
    \item The Treaty of Alliance with France was formed in 1778
    \item Spain declared war against Great Britain in 1779 
    \item Dutch Republic entered the war against Great Britain in 1780 \cite[85,89]{stewart_2005}
  \end{itemize}
  &
  \begin{itemize}[nolistsep,leftmargin=*]
      \item  Did not have allies
      \item  Several German states provided troops of
	mercenaries \cite[62]{stewart_2005}
      \item  Retained loyalty of some colonists
  \end{itemize}\ML
  \multicolumn{2}{c}{Informational}\ML
	\begin{itemize}[nolistsep,leftmargin=*]
	    \item  Colonies published information in 27 newspapers; pamphlets
	      etc. \cite[49,199]{knollenberg_growth_2003}
	    \item  They used propaganda.  Examples: ``Farmer's Letters'' by John
	      Dickinson published in 1767-1768 throughout the colonies; “Boston
	      Massacre” engraving by Paul Revere, and public orations from 1771
	      to 1775; the pamphlet “Common Sense” by Thomas Paine in 1776;
	      Liberty poles
	      erected \cite[3,48-53,81,227]{knollenberg_growth_2003} \cite[101]{ladenburg_causes_1989}
	    \item  Communication was well established, and many letters were
	      used as means of communication \cite[167,264]{knollenberg_growth_2003}
	    \item  They did not have many difficulties in providing information
	      about the enemy \cite[63]{ladenburg_causes_1989}
	\end{itemize}
	&
	\begin{itemize}[nolistsep,leftmargin=*]
	    \item  Used publications throughout British Empire
	    \item  Used propaganda, like the printing of  “A New Method of
	      Macaroni Making as Practiced at Boston in North America”
	    \item  Sometimes the propaganda was confusing, and could be
	      interpreted as counter-propaganda, like the printings “The
	      Bostonians in Distress”, or “The Wise Men of Gotham and Their
	      Goose” \cite[168-169]{knollenberg_growth_2003}
	    \item  Letters were used as means of communication \cite[169]{knollenberg_growth_2003}
	    \item  Communication faced problems of great distance from Great Britain and hostile territory
	    \item  Had difficulties in providing intelligence information \cite[545]{mackesy1962british}
	\end{itemize}\ML
	\multicolumn{2}{c}{Military}\ML
	\begin{itemize}[nolistsep,leftmargin=*]
	    \item  Had significant allies in European great powers: France,
	      Spain, and Dutch Republic \cite[p.89]{stewart_2005} 
	    \item  At the beginning of the war had militia as armed forces for
	      local defense \cite[p.56]{stewart_2005} 
	    \item  People from militia were not used to fighting on a wide area
	      \cite[p.48]{pancake_1985} 
	    \item  Organized and trained companies of militia for rapid response
	      called minutemen \cite[p.46]{stewart_2005} 
	    \item  Formed Continental Army, but militia men were constantly
	      filling the gaps \cite[p.53]{stewart_2005} 
	    \item  The leaders had experience in waging war, but more as
	      tactical than strategic leaders
	      \cite[7]{higginbotham_daniel_1961} \cite[p.51]{stewart_2005} 
	    \item  Officers were elected \cite[p.51]{pancake_1985}, but the Continental Army faced the problem of not having enough men competent to be officers 
	    \item  The soldiers were lacking training and discipline
	      \cite[p.52]{pancake_1985} 
	    \item  People who were wealthier than the others avoided recruitment
	      \cite[p.26]{stephenson_patriot_2007}, and even the criminals were
	      recruited \cite[p.27]{stephenson_patriot_2007} 
	    \item  People were deserting from the military forces
	      \cite[p.48]{pancake_1985} 
	    \item  The number of men engaged in the war was decreasing because
	      of diseases \cite[p. 33]{stephenson_patriot_2007} 
	    \item  Had problems with supplies
	      \cite[p.105]{stephenson_patriot_2007}
	\end{itemize}
	&
	\begin{itemize}[nolistsep,leftmargin=*]
	    \item  Didn’t have allies, but German mercenaries, and American
	      Loyalists were fighting on their side
	      \cite[p.133]{higginbotham_daniel_1961} 
	    \item  Had standing armed forces
	      \cite[p.48]{higginbotham_daniel_1961} 
	    \item  British forces were far from home 
	    \item  Had the greatest naval power in the world
	      \cite[p.43]{stephenson_patriot_2007} 
	    \item  Had problems with the choice of officers
	      \cite[p.543-544]{mackesy1962british} 
	    \item  The soldiers were well-trained and well-disciplined
	      \cite[p.42]{pancake_1985} 
	    \item  They did not fight for 15 years prior the war and some of
	      them were inexperienced \cite[p.44-45]{stephenson_patriot_2007} 
	    \item  Most of the recruited were poor
	      \cite[p.38]{stephenson_patriot_2007}, many of them criminals
	      \cite[p.36]{stephenson_patriot_2007} 
	    \item  Faced the problem of deserting of men from the military
	      forces \cite[p.48]{pancake_1985} 
	    \item  The number of men engaged in the war was decreasing because
	      of diseases \cite[p.62]{stewart_2005} 
	    \item  They held the strategic initiative
	      \cite[p.540]{mackesy1962british} 
	    \item  Had problems with supplies, especially because of the
	      transport \cite[p.541]{mackesy1962british}\cite[62]{stewart_2005} 
	    \item  They could not get reinforcements immediately if they needed it
	\end{itemize}\ML
	\multicolumn{2}{c}{Economic}\ML
	\begin{itemize}[nolistsep,leftmargin=*]
	     \item Had economic growth before the Revolution
	       \cite[77,91]{ladenburg_causes_1989} but at the time of Revolution the growth declined
	     \item In 1777, Congress had a debt of \$20,000,000
	       \cite[p.102]{stephenson_patriot_2007}
	       \cite[34]{higginbotham_daniel_1961}
	     \item Manufacturing was restricted
	       \cite[p.76]{ladenburg_causes_1989}
	     \item Most of the economy was based on plantations, hunting,
	       fishing, exploring the forests \cite[p.75]{ladenburg_causes_1989} 
	     \item They were restricted in trading by Great Britain, but many
	       were not obeying the laws; they even negotiated secretly for
	       trading \cite[p.229]{higginbotham_daniel_1961}
	     \item Issued paper money but its value  was dropping during the war
	       \cite[p.27]{stephenson_patriot_2007}
	     \item Did not have efficient federal tax system to collect money 
	     \item France helped American economy by loaning large sums of money
	       \cite[p.233]{higginbotham_daniel_1961}
	\end{itemize}
	&
	\begin{itemize}[nolistsep,leftmargin=*]
	    \item Had a national debt of £140,000,000 after the French and
	      Indian War\cite[p.88]{ladenburg_causes_1989} 
	    \item Industrial Revolution had begun \cite[p.37]{pancake_1985}, but
	      people were losing their jobs \cite[p.37]{stephenson_patriot_2007} 
	    \item Explored their colonies to increase their wealth
	      \cite[p.75]{ladenburg_causes_1989} 
	    \item Established trade regulations to protect their economy
	      \cite[p.76]{ladenburg_causes_1989} 
	    \item Had banks, national currency, and well established paying systems 
	    \item Had tax system that collected money 
	\end{itemize}\ML
      \end{mpxtabular}
\end{singlespace}
%\end{table}
%\end{document}
