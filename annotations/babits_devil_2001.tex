\red{Alvarado}
The author is a professor of maritime history and nautical archaeology at East
Carolina University in Greenville, North Carolina. The target audience for this
read is anyone who is interested in discovering more about the Battle of
Cowpens, in a sense that possibly no other author has dedicated as much
attention to this single battle. Babits identifies the immense importance of
the battle, identifies each force (leadership, equipment, training), and walks
through the battle in detail.

\red{Nicholson}
The author is an American archaeologist and is credited for his
extremely accurate accounts of Soldiers' combat experiences in the18th century.
He is a professor of Maritime Archaeology and History at East Carolina
University. I found this book to be unbiased.  This book was useful in
determining the operational situation.  I highly recommend this source.   

\red{Flores}
Babits is credited for having such detail in this book. Devil of a whipping won
a Distinguished Book Award in 1998. Babits says he draws his experience of war
soldier life through his re-enactments from participating in groups portraying
Revolutionary Continental Soldiers from 1967-84. Aside from his re-enactments
Babits is also an archaeologist Babits is currently a professor at East
Carolina University.  This book was useful for the detail that Babits went into
before during and after the Battle of Cowpens.

\red{Futch}
The author is an American archaeologist and is credited for his extremely
accurate accounts of Soldiers' combat experiences in the 18th century.  He is a
professor of Maritime Archaeology and History at East Carolina University.
This publication was useful in detailing the tactics of each side, as well as
describing the physical characteristics of the battlefield.  The author, in
using a wide range of sources, was able to piece together a cohesive story of
what occurred at Cowpens.
