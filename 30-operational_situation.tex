\section{The Operational Situation}

This section will review the Operational Situation of the Battle of Cowpens.
It will place the battle within the context of the Southern Campaign.   For the
purposes of this section, the campaign began with the British forces invading
South Carolina in February 1780 and this section will conclude January 16, 1781
just prior to the Battle of Cowpens.

The British plan of the Southern Campaign and its objectives was conceived by
Lord Germain and embraced by Sir Henry Clinton.  First, the British Army was to
seize Charleston.  Next, it would raise Loyalist militia forces to gain control
of South Carolina and then North Carolina.  Finally, they were to destroy the
Continental Army of the Southern Department and end all forms of rebellion.

In order to achieve it's Southern Campaign objectives, the British Army
required the raising of Loyalist militias.  The idea was for the British Army
to move from the coast westward gaining control of new areas.  As they gained
control of an area, Loyalists militias would be responsible for maintaining
security and lines of communication as the British Army progressed.  From
lessons learned in the North, ``the British had discovered that Loyalists would
only muster when regular British forces were on hand to protect them''
\cite[43]{woodward_comparative_2002}.

In February of 1780, General Henry Clinton's force consisting of 10,000
soldiers, supported by 5,000 sailors landed off Charleston, South Carolina
\cite[6]{weigley_partisan_1970}.   In response, Major General Benjamin Lincoln set up a defense for
the city of Charleston. General Lincoln's force consisted of about 2,650
Continentals and 2,500 South Carolina militia \cite[6]{weigley_partisan_1970}.  Lincoln's force was
inadequate for the defense, his decision to stay and defend may have been more
politically motivated than tactically thought out.  On 12 May 1780, General
Lincoln surrendered after a prolonged siege.   Virtually all the ``organized
military strength'' in South Carolina had been wiped out.  The seizing of
Charleston allowed the British Army to achieve their first campaign objective
almost immediately.  Charleston allowed for a ``good port and base from which
to mount their Southern Campaign''. \cite[22]{woodward_comparative_2002}

British Lieutenant Colonel Banastre Tarleton wiped out the remaining American
forces in South Carolina at the Waxhaws settlement.  Tarleton's force beat 350
Virginia Continentals under Colonel Abraham Buford's command and a small
cavalry unit of Lieutenant Colonel William Washington \cite[7]{weigley_partisan_1970}.  Waxhaws is
where Tarleton earned the nickname ``Bloody Tarleton''. \cite[20]{moncure_cowpens_1996}

The British army ``established garrisons at Georgetown and Beaufort as well as
the capital along the coast, and at Camden, Rocky Mount, and Ninety Six to
shield the northern border of the province and watch over the interior''
\cite[10]{weigley_partisan_1970}.  Settlements in the far west put the British in contact with the
``friendly, anti-Revolutionary Creek and Cherokee Indians and threatened
American settlements beyond the mountains'' \cite[10]{weigley_partisan_1970}

In June 1780, General Clinton returned to New York leaving General Earl
Cornwallis in Command.  General Cornwallis was a proven British field commander
with a wealth of experience in the American area of operations.  General
Cornwallis commanded a force  of about 3,000 soldiers \cite[51]{woodward_comparative_2002}.  With this
relatively small force, he was to achieve the Southern Campaign objectives
given to him by General Clinton.   

General Clinton gave Cornwallis instructions to defend Charleston above all
else because of its logistical importance and political value.  He was to
maintain gains in Georgia and establish control over South Carolina eventually
moving into North Carolina.  Clinton also expected Cornwallis ``to provide some
3,000 troops for operations in the Chesapeake region at some indefinite time in
the future'' \cite[51]{woodward_comparative_2002}.

In order to achieve these objectives Cornwallis would have to raise Loyalist
militias.  The British Army required the use of Loyalist militias as a means to
reach the Southern Campaign Objectives.  Even with sizable Loyalist militias it
would be nearly impossible to achieve all of the Southern Campaign objectives
with a force of 3,000 troops augmented by Loyalist militias.   In order to have
achieved the campaign objectives listed above, the British needed to win the
hearts and minds of the people to even have a chance of being successful.  The
British consistently alienated the neutral American or pro-British citizenry
through failed policies forcing these non players into opponents.

The British exercised poor decision making when dealing with the treatment of
the American populations following the fall of Charleston.  An example of
failed policy when dealing with paroled prisoners from American militias came
just prior to General Clinton's handover to General Cornwallis.  The following
passage describes the change in policy: 

\begin{quote} ``Just before departing Clinton announced abandonment of the
  system whereby rebels were regarded as under parole and substituted a policy
  which offered full civil rights to all who showed complete loyalty, but
  punishment as enemies to any who did not; this compulsion of a clear choice
  between loyalty and a return to rebellion, when many would have preferred a
  neutrality which might have served Clinton just as well, perversely helped
  drive men back into rebellion'' \cite[12]{weigley_partisan_1970}. 
\end{quote}

This British policy for dealing with prisoners on parole unnecessarily
increased the combat power of the Americans.  

Another example of failed British policy was that British leaders allowed their
troops to loot and commit crimes against civilians.  This drove otherwise
neutrals to side with the ``Patriot cause'' \cite[13]{weigley_partisan_1970}.
The Scotch-Irish of the ``upcountry'' were ``driven into rebels arms'' by a
British policy allowing hostility towards the Presbyterian church
\cite[13]{weigley_partisan_1970}.  Presbyterian churches were burned simply
because they were thought to harbor rebels leaders. 

The upcountry of South Carolina was the frontier and contained a disproportion
number of Loyalists.  Upcountry folks distrusted ``anything the low country
did; and because the lowland planters led the Revolutionary movement, the
upcountry tended to distrust the Revolution as an event calculated simply to
render them still more subservient to the planter's political power.  By the
time the war began, about half the population of South Carolina resided in the
upcountry, though with much less than half the membership of the assembly
``\cite[11]{weigley_partisan_1970}.

The disproportionate number of Loyalists in the upcountry helped inspire the
British concept of the Southern Campaign.  Instead of exploiting this wedge and
using the inlanders to fill the ranks of Loyalist militias in order to reach
British campaign objectives, failed British policy forced these neutral or
pro-British populations to rise against the British.  Native Americans were
also poorly integrated into British strategy to achieve campaign objectives.
Before the Southern Campaign, the Cherokee also opposed the Revolution, but
they did ``not coordinate their actions with either the Loyalists or the
British, and in 1776, Williamson led an expedition against them which
considerably stunted their ability to make trouble''
\cite[12]{weigley_partisan_1970}.

After the fall of Charleston many of these failed British policies led to a
rising resistance.  Many alienated groups as well as those already opposing
British rule began taking up arms.  However, the resistance was poorly planned
and organized with no real campaign objectives in mind except  ``self
preservation and revenge'' \cite[14]{weigley_partisan_1970}.  The pockets of
resistance utilized guerilla warfare to fight the British.  

Thomas Sumter ``the Gamecock'' and other rebels moved on horseback moving much
too fast for larger British Forces to catch them.  They relied on surprise and
guerilla warfare like tactics.  These terrorist attacks forced the British to
``disperse their forces over wider and wider areas and thus made them still
more vulnerable'' \cite[16]{weigley_partisan_1970}.

British failed policy gave way to a strengthened informal resistance further
occupying and weakening British Forces.  This allowed the Continental Congress
to re-establish a more traditional Army in the Southern Department.

In July 1780 Major General Horatio Gates took command of the Southern
Department due to Major General Baron Jean De Kalb being passed over, because
he was foreign born.  The command consisted of about 4,200 Continentals and
militia \cite[21]{moncure_cowpens_1996}.  This force was adequate to annoy and
harass the British but not destroy them.  Instead of utilizing his forces in
this manner, he returned to traditional warfare before his force was ready and
trained.  His first objective was to take Camden, with a direct straight
advance to the city. 

On August 16, 1780 the Continental Army met the British Army in an open field
in Camden.  The British had more than 2,000 men and Gates had about 3,000
\cite[19]{weigley_partisan_1970}.  Quantitatively the Americans were not
outmatched, but qualitatively the British had a significant advantage.  The
British won the battle quite easily with Gates leading the retreat with about
700 Americans surviving the battle \cite[21]{weigley_partisan_1970}.   De Kalb
died with the ``Maryland and Delaware Continentals, still leading them in a
counterattack after everyone else had fled'' \cite[21]{weigley_partisan_1970}.
After Camden, Cornwallis prepared to move into North Carolina by ``sending word
to Loyalists in that state to begin assembling and seizing Patriot stores''
\cite[27]{woodward_comparative_2002}.

Virtually all formal organized Continental forces were destroyed leaving only
the informal resistance to survive.  Once again, these pockets of resistance
utilized guerilla warfare to annoy and harass the enemy quite successfully.  A
leader of one of these resistance groups was Colonel Francis Marion.  Marion
``the Swamp Fox'' and his force of disreputables rose from the coastal swamps
and caused enough trouble to cause Tarleton to pursue him.  Tarleton found
pursuing Marion so useless that he decided to pursue Sumter instead.  Tarleton
was supposed to have said, ``Let us go back and we will find the gamecock.  But
as for this damned old fox, the devil himself could not catch him!'' \cite[23]{weigley_partisan_1970}

Another example of failed British policy was the treatment of neutral civilian
population at Kings Mountain. The negative treatment drove them to become part
of the rebellion against British forces when no organized Southern Continental
forces existed,.  Major Patrick Ferguson of the 71st Highlanders ``threatened
the settlements beyond the Appalachians that they had better make peace or be
punished'' \cite[24]{weigley_partisan_1970},\cite[22]{moncure_cowpens_1996}.
His threats motivated 1400-1800 mountain men to rise up in arms against him.
The Mountain men ``killed Ferguson and 157 of his Loyalists and captured the
rest, 163 badly wounded and 698 others'' \cite[25]{weigley_partisan_1970}. 

Tarleton and ``the Gamecock'' met head to head for a final time at Blackstocks
Plantation on 20 November 1780, on the Tyger River.  Both sides fought to a
stalemate and slipped away at night claiming victory.  Sumter was wounded
during the encounter, no longer being a significant force in the Southern
Campaign. 

The informal resistance groups kept the British occupied, allowing the
Continentals to once again raise an Army.  On December 2, 1780 Greene arrived
at Charlotte, North Carolina to assume command
\cite[22]{moncure_cowpens_1996},\cite[25]{weigley_partisan_1970}.  His combat
experience included ``participation in every major battle under Washington
between 1776 and 1779''. \cite[5]{babits_devil_2001}.  Greene assumed command
of a force ``2,457 strong, though with only 1,482 present and fit for duty,
only 1,099 of the muster roll strength Continentals''
\cite[27]{weigley_partisan_1970}.   However, Greene described his force as
being ``wretched'' and ``distressed'' and ``starving with cold and hunger,
without tents and camp equipage'' \cite[23]{moncure_cowpens_1996}.

American strategy during the Southern Campaign was ``disjointed''
\cite[60]{woodward_comparative_2002}. This was caused by a ``series of
leadership turnovers that produced frequent changes to the strategic
direction'' \cite[61]{woodward_comparative_2002}.

\begin{quote}
``lack of continuity was exacerbated by a severe shortage of resources to
conduct the campaign with.  The Southern department received little external
assistance.  Each successive American commander had to figure out for himself
how to mount an effort that would support the national military effort''
\cite[61]{woodward_comparative_2002}
\end{quote}

Washington's primary Strategic goal for the Southern Campaign was to break the
British will to fight.  Lincoln and Gates both pursued strategies that were
destined to fail to reach their campaign objectives, thus failing to accomplish
Washington's strategic goal.  Both Lincoln and Gates pursued a conventional
campaign by seeking to engage the British in decisive battle.  Lincoln's goal
was to prevent the British a ``foothold in the South'' and Gates' goal was to
push the British ``back to the sea'' \cite[64]{woodward_comparative_2002}.

Nathanael Greene's time in command covered more time in the Southern campaign
than any other Commander of the Southern Department.  For this reason and his
overall success, his strategy and campaign objectives are the clearest during
the Southern Campaign.  General Washington's only guidance to Nathanael Greene
came by way of this statement, ``Uninformed as I am of the enemy's force in
that quarter, of our own, or of the resources, I can give you no particular
instructions, but must leave you to govern yourself entirely according to your
own prudence and judgement, and the circumstances in which you find yourself''
\cite[63]{woodward_comparative_2002}.

Greene's campaign objectives were to avoid destruction of his force, raise
militia forces, and wear down Cornwallis.  However, he had to address
logistical concerns before he could begin to hope to reach his operational
objectives.   General Greene knew that if his army was destroyed it would leave
the Carolinas without a formal Southern Department of the Continental Army for
a third time.   This possible destruction may have also resulted in another
force being unable to be raised.  His primary responsibility as he saw it, was
to be conservative and maintain his Army at all costs without taking great
risks. Nathanael Greene would not repeat the mistakes of his predecessors by
fighting the British on their terms.   Greene knew that the ``strategic value
of his army was in its existence more than its ability to defeat its British
counterpart''. \cite[66]{woodward_comparative_2002}

Greene divided his army by moving his main force to Hick's Creek (Cheraws),
South Carolina, sending General Morgan west and Lieutenant Colonel Henry east
to join Marion with his operations on the coast.  The combined forces of Lee
and Marion attacked Georgetown, South Carolina, on 25 January 1781.  This
division of forces was due mainly for logistical reasons.  Colonel Tadeusz
Kosciuszko had been sent in search of supplies and found ``adequate supplies''
in Cheraws for the main body
\cite[23]{moncure_cowpens_1996}\cite[27]{weigley_partisan_1970}.   By dividing
his forces it lessened the number of soldiers in one place thus requiring less
supplies needed from each location. 

Brigadier General Daniel Morgan led the smaller force of 600 west towards
Ninety Six \cite[27]{weigley_partisan_1970}.  Greene's orders to Morgan were to
conduct operations, ``either offensively or defensively, as your own prudence
and discretion may direct-acting with caution and avoiding surprises by every
possible precaution'''' \cite[27]{weigley_partisan_1970}.  On December 16,
1780, Morgan received further guidance from Greene to conduct a ``prudent
campaign designed to call attention to itself''
\cite[24]{moncure_cowpens_1996}.  The guidance to Morgan was to raise a
militia, protect patriot settlements, improve moral of the people and ``annoy
the enemy in that quarter \cite[24]{moncure_cowpens_1996}.  Finally, if
Cornwallis were to pursue Morgan he was to rejoin the main army.

Morgan left Charlotte on December 21, marching about 55 miles and encamped at
Grindal Shoals on Christmas Day on the Pacolet River.  Following Greene's
guidance, Morgan raised militia forces augmenting his detachment to about 1,040
men \cite[28]{weigley_partisan_1970}.   The force included ``320 of the stout Maryland and Delaware
Continentals and 60 to 100 Continental light dragoons, who could fight either
mounted or on foot, under William Washington'' \cite[28]{weigley_partisan_1970}.   On December 28,
Washington's Continental dragoons and militia numbering about 300 destroyed a
Tory force of 150 at Hammond's Store then moved south and burned Fort William a
short distance from Ninety Six
\cite[8]{babits_devil_2001}\cite[24]{moncure_cowpens_1996}.

Cornwallis could not direct his army toward either smaller Continental force
without exposing Charleston to attack or leaving the Western region vulnerable.
In response to Greene's division of forces, Cornwallis was forced to divide his
own force, sending Tarleton with about 1,100 men, mixed cavalry and infantry to
deal with Morgan \cite[28]{weigley_partisan_1970}.   Tarleton moved westward and placed his force
between Morgan and Ninety Six.  Tarleton then moved his forces northward toward
Morgan's encampment at Grindal Shoals.   This move forced Morgan to move his
force North with Tarleton in pursuit.  

Tarleton aggressively pursued Morgan to an area known as Cowpens.  This area is
where these two forces would soon meet on the field of battle.  The Battle of
Cowpens itself will be discussed further in the following sections of this
paper.  The battle proved to be a turning point in the Southern Campaign of the
Carolinas in favor of the Americans.

The British Army from the beginning of the Southern Campaign, planned and
prepared to conduct major combat operations rather than irregular warfare.
According to Army Field Manual 3-0 Operations 2-12, an ``operational theme
describes the character of the dominant major operations being conducted at any
time within a land force commander's area of operations.''  The two dominant
operational themes throughout the Southern Campaign for both sides were major
combat operations and irregular warfare.

In the beginning of the Southern Campaign at Charleston and later at Camden,
both sides followed the operational theme of conducting major combat
operations.  Major combat operations can be simply described as army versus
army.  The Army Field Manual 3-0, 2-69, describes successful major combat
operations as being able to: 

\begin{quote}
``\dots defeat or destroy the enemy's armed forces and seize terrain.
Commanders assess them in terms of numbers of military units destroyed or
rendered combat ineffective, the level of enemy resolve, and the terrain
objectives seized or secured. Major combat operations are the operational theme
for which doctrine, including the principles of war, was originally
developed.''
\end{quote}

This operational theme describes why at the onset of the Southern Campaign and
later after Camden the Americans seemed destined to lose the campaign and
eventually the War for Independence.  The British had a marked advantage both
in quality and quantity for major combat operations.

The periods following the fall of Charleston and later Camden and to a lesser
degree throughout the entire Southern Campaign, the operational theme shifted
from major combat operations to irregular warfare for the Americans.  According
to FM 3-0, 2-45, irregular warfare ``is a violent struggle among state and non
state actors for legitimacy and influence over a population''.  The Field
Manual further describes irregular warfare as being amongst the people and
lacking the goal of military supremacy.  The FM 3-0, 2-46 also describes
irregular warfare as avoiding,

\begin{quote}
 ``\ldots a direct military confrontation. Instead, it combines irregular
 forces and indirect, unconventional methods (such as terrorism) to subvert and
 exhaust the opponent. It is often the only practical means for a weaker
 opponent to engage a powerful military force. Irregular warfare seeks to
 defeat the opponent's will through steady attrition and constant low-level
 pressure.''
\end{quote}

After General Greene assumed command of the Continental Army the operational
themes combined elements of both major combat operations as well as irregular
warfare, with an emphasis being on irregular warfare.  Irregular warfare was
conducted by pockets of resistance still conducting terrorist and insurgent
activities while Greene's use of Morgan raising and training militias fell more
closely under unconventional warfare.  In addition, Green still used Morgan's
force in more conventional means that would be classified as major combat
operations, such as Hammond's Store and later at Cowpens.

For the most part, the British conducted operations that combined elements of
both major combat operations as well as irregular warfare, but focused
primarily on major combat operations while irregular warfare fell to the
wayside.  The British did conduct counter insurgency and counter terrorism
activities but usually with horrible consequences due to a lack of British
policy towards the American people, ultimately losing the battle for their
hearts and minds.  The focus by Cornwallis was major combat operations
illustrated by his movement of forces in pursuit of the next decisive battle.  

Ultimately, Nathanael Greene's use of the combined operational themes of major
combat operations and irregular warfare, allowed him to achieve his Southern
Campaign operational objectives.  While the British Army mainly employed major
combat operations with an inadequate force that was ``overburdened by fighting
a conventional war and an unconventional war simultaneously''
\cite[55]{woodward_comparative_2002}.  This failed strategy by Cornwallis,
resulted in his Southern Campaign operational objectives not being met.


\subsection{The Context Within the Campaign}

\subsection{Operational Themes (FM 3-0)}

\subsection{Campaign Objectives}

\subsection{Operational Events Leading to the Battle}

\ctable[
  cap={Events leading to the Battle},
  caption={Operational events leading to the Battle of Cowpens.},
  pos=h,
  ]{>{\footnotesize}l>{\footnotesize}p{4in}}{}{							\FL
Year 		& Event 									\ML
~		& ~~~~~~~~~~~~~~~~~~~~~~~~~~~~~~~~~~~~~~~~~~~~~~				\NN
~ 		& ~ 										\LL
}

%---------------------
% % http://militaryhistory.about.com/od/americanrevolution/a/amrevcauses.htm
% Rise of Liberalism and Republicanism
% 
% As tensions regarding colonial lands and taxation increased during the 1760s
% and 1770s, many American leaders were influenced by the liberal and republican
% ideals espoused by Enlightenment writers such as John Locke. Key among Locke's
% theories was that of the ``social contract'' which stated that legitimate state
% authority must be derived from the consent of the governed. Also, that should
% the government abuse the rights of the governed, it was the natural
% responsibility of the people to rise up and overthrow their leaders. The ideas
% of Locke and other similar writers contributed to the American embrace of
% ``republican'' ideology in the years before the Revolution. Standing in
% opposition to tyrants, republicanism called for the protection liberty through
% the rule of law and civic virtue.
%--------------------------




