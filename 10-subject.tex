\section{Introduction}

  %\red{\emph{DEFINE THE SUBJECT [Begin your battle analysis by determining the
  %date, location, and who the opponents were. Look at what sources are available
  %that describe the operation and determine how you will gather information. For
  %this exercise, all your sources are on Blackboard, AKO, Woodworth Library, or
  %DOD Military Websites.]}}
  %

  %\begin{singlespace}
  % \red{\small\emph{Determine the date, location and principal antagonists.\\
  % When did the battle occur?\\
  % Where did it take place?\\
  % Who was involved?}}
  %\end{singlespace}

Colonial and British forces fought the Battle of Cowpens, at Hannah's Cowpens,
South Carolina, as a late battle in the Southern Campaign of the American
Revolution on January 17, 1781. At the end of Lt. Col Banastre Tarleton's
pursuit, Brig. Gen Daniel Morgan chose Hannah's Cowpens as the
battlefield to halt Tarleton and his streak of victories. Ultimately, this
turned the tide of the war.

This paper describes the Battle of Cowpens. First we will discuss our sources, followed
by the campaign, the battle itsself, and its ultimate significance.

  % The Battle of Cowpens, Jan. 17, 1781.
  %  WM Washington's Calvary
  %  Morgan's Continentals
  %  Pickens' Militia

  % Tarleton
  %

  %Colonists has \red{...} regular and militia forces organized in \red{...} under
  %\red{General ...}. They had been conducting operations across South Carolina... 
  %
  %The British under General \red{...} consisted of \red{...} forces. 



%\subsection{Discussion of Sources}

  % \begin{singlespace}
  % \red{\small\emph{(1) What are the sources of information concerning the battle?\\
  % (2) What types of sources are required for a thorough, balanced account of the fight?\\
  % (Biographies, operational histories, battle journals, after-action reports, war diaries, etc.)}}
  % \end{singlespace}

Several sources of information are available for the study of Cowpens. These
include historical works, writings and memoirs from primary sources, maps, state
and national park service documents and the battlefield terrain itself, still
preserved as the Cowpens' National Battlefield near Chesnee, SC. Since some
writers will focus on specific aspects of the battle, or on either the British
or Colonial forces, the variety of sources we pull from provides balance in our
analysis. Below is a listing of each reference used in our battle analysis,
with specific analysis towards each author's background, perspective, focus and
finally our recommendations towards use of each source.

\begin{singlespace}
  %\subsection{Annotated Bibliography}
  % \red{\small\emph{(1) Who wrote the book?\\
  % (2) What was the writer's point of view or bias?\\
  % (3) What areas of your assigned battle does the source provide the most help for research?\\
  % (4) Do you recommend this book for use by others?}}
  %\vspace{-2em}
\nocite{*}
\bibliographystyle{jurabib}
\bibliography{99-references}
\end{singlespace}
