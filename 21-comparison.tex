\subsection{Comparison of the Antagonists}

% Overview text goes here.
% 
\ctable[
	cap = Comparison of Antagonists,
	caption = Comparison of US and Colonial Forces,
	]{p{2.5in}p{2in}p{2in}}{
	\tnote[a]{note a}}
	{\FL%\small
\multirow{2}{*}{Elements of National Power}	&	\multicolumn{2}{c}{Antagonists} \ML
						&	Colonies	& Britain	\ML
Diplomatic			&
	The Treaty of Alliance with France was formed in 1778.\newline
	Spain declared war against Great Britain in 1779\newline
	Dutch Republic entered the war against Great Britain in 1780 (Stewart,
	p.85, 89)
	& Did not have allies\newline
	Several German states provided troops of mercenaries (Stewart,
	p.62)\newline
	Retained loyalty of some colonists
	\NN
%
%C2 / IO				&
%	- Colonies published information in 27 newspapers; pamphlets etc.
%	(Knollenberg, p.49, 199)  
%
%	- They used propaganda.  Examples: “Farmer’s Letters” by John Dickinson
%	published in 1767-1768 throughout the colonies; “Boston Massacre”
%	engraving by Paul Revere, and public orations from 1771 to 1775; the
%	pamphlet “Common Sense” by Thomas Paine in 1776; Liberty poles erected
%	(Knollenberg,p.3, 48-53, 81, 227; Ladenburg, p.101)
%
%	- Communication was well established, and many letters were used as
%	means of communication (Knollenberg, p.167, 264)
%
%	- They did not have many difficulties in providing information about the
%	enemy (Ladenburg, p. 63)
%	&
%	- Used publications throughout British Empire
%
%	- Used propaganda, like the printing of  “A New Method of Macaroni
%	Making as Practiced at Boston in North America”
%
%	- Sometimes the propaganda was confusing, and could be interpreted as
%	counter-propaganda, like the printings “The Bostonians in Distress”, or
%	“The Wise Men of Gotham and Their Goose” (Knollenberg, p.168-169)
%
%	- Letters were used as means of communication (Knollenberg, p.169)
%
%	- Communication faced problems of great distance from Great Britain and
%	hostile territory
%
%	- Had difficulties in providing intelligence information (Mackesy, p.
%	545)
%	\NN
%Factor				&	Colonies	& British	\\\midrule
%National (Strategic) objectives	&			&		\\
%Insturments of National Power	&			&		\\
%Military Systems		&			&		\\
%Previous performance		& French and Indian War	&		\\\bottomrule
	}



\subsubsection{National (Strategic) objectives}

\index{Strategic Objectives}

\paragraph{British}

Strategic Objectives: ``maintain the thirteen colonies as a British
possession''\cite[2]{moncure_cowpens_1996}

``British military objectives were fourfold: separate the New England colonies
from the others by seizing the Hudson River north to Lake Champlain; isolate the
'bread basket' colonies of Pennsylvania and Maryland; control the southern
populace by holding Charleston, Georgetown, and the line of the Santee River;
and, finally, blockade the entire American coast to prevent an influx of arms
from abroad.''\cite[2]{moncure_cowpens_1996}

Strategic Strategies: ``divide the Colonies by applying economic sanctions to
the most rebellious''\cite[2]{moncure_cowpens_1996}

%Other British National objectives: other colonies, West Indies, trade routes,
%(discuss mercantilism?), 

%``\ldots d'Estaing's presence in the Caribbean together with a Spanish fleet in
%Havana (after Spain's entry on the American side in June 1779) forced Clinton to
%send 8,000 troops to the West Indies. These transfers weakened Clinton so
%severely that he never seriously challenged Washington's position at West Point,
%New York.''\footcite[10]{moncure_cowpens_1996} \index{d'Estaing}

\paragraph{Colonies}

\ldots

\subsubsection{Instruments of National Power}

%\begin{table}
%  \begin{center}
%  \begin{tabular}{lll}\toprule
%    Instrument & British & American \\\midrule
%	a & a & a \\\bottomrule
%  \end{tabular}
%  \end{center}
%  \caption{Comparison of British and Colonial instruments of National Power.}
%\end{table}

\ldots

\subsubsection{Military Systems}

\ldots

\subsubsection{Previous performance}

\ldots
